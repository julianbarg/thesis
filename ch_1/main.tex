\documentclass[12pt, man, natbib]{apa6}
\usepackage[USenglish]{babel}
\usepackage{setspace}
\usepackage{hyperref}

\title{All for Naught in the Pipeline Industry? When Learning Does Not Solve the Problem}
\shorttitle{All for Naught in the Pipeline Industry?}
\author{Julian Barg\\barg.julian@gmail.com}
\affiliation{Ivey Business School}
\setcitestyle{authoryear, open={()},close={)},citesep={,},aysep=}

\abstract{Pipeline spills are both frequent and serious pollution events. Since the 1990s, the industry has made great strides in developing computerized equipment to prevent and control leaks. Yet, pipeline spills today are almost as prevalent as they were in the 1980s, raising the question of why technology does not allow us to overcome the problem of normal accidents, even in simple systems. This paper juxtaposes learning in the pipeline industry on the incident and population level. The results indicate that broad, sweeping technological changes can miss the mark. Social processes, including the regulator that is supposed to act as a watchdog, can function to mask this problem by promoting the technological "solution".}

\begin{document}
	
	\maketitle
	
	\singlespacing
	
	\section{}
	
	Organizational learning comes down to choices. Firms can either invest in improving existing technology, or develop new technology \citep{March1991}. Investing in the "wrong" technology can lead to technological lock-ins \citep{Levinthal1993}. The actors in the pipeline industry have selected a number of technological solutions to resolve their most pressing issues. In terms of direct, regular environmental impacts, the industry is performing well. Transporting liquids by pipeline (or by pipe in general for that matter) is much more efficient than the alternatives, transport by rail or truck. But the industry is regularly shook by spills. When a pipeline spill occurs, the oil quickly infiltrates the soil and seeps into the groundwater.\footnote{The infiltration depth in sand is assumed to be over 10m in the first day alone \citep{Bonvicini2015}.} The environmental degradation caused by oil affects the local environment, and the local populace, too: a 2019 sibling comparison study on oil spills in Nigeria found that in localities that are affected by oil spills, for every 1,000 live births, an additional 38.3 neonatal deaths occur\citep{Bruederle2019}.
	
	The most prominent technologies of pipeline operators in their fight against pipeline spills are smart pigs, leak detection systems, and SCADA systems. Smart pigs get their name from the screeching sound they make when they move through a pipeline. These devices measure utilize electromagnetic flux or ultrasonic probing to assess corrosion or mechanical damages to the pipe \citep{Singh2017-7}. Leak detection systems can be broken down into internal and external systems. Internal systems generally measure the flow of oil at two points A and B to detect any loss in between those points. External systems are external sensors that detect signs of escaping hydrocarbons, such as acoustic, hydrocarbon, or temperature sensors \citep{Shaw2012}. Finally, SCADA systems are systems that allow an operator remotely monitor and operate lines. The operator typically sees on his screen charts of the flow at different points, can open and close valves, and startup or shutdown delivery of oil. Alarms from leak detection systems of the line are also displayed to the SCADA operator.\footnote{Larger pipeline companies operate control centers where all lines in a region are managed. Operators usually operate multiple SCADA systems at once, and more experienced employees supervise the operators. Control centers are operated in formal hierarchy, where for certain operations (such as clearing an alarm), a SCADA operator will require the go-ahead from a supervisor. See \citet{NTSB2012} for an in-depth description of an Enbridge control center in Edmonton as of 2012.}

	The high technology character of leak detection often stands in stark contrast to the experienced reality of oil spills. A 2012 study commissioned by the Pipeline and Hazardous Materials Safety Administration (PHMSA) of oil spills that occured over a 19 month period, SCADA systems assisted in less than 25\% of cases with the detection and confirmation of the spill \citep[p. 3-33]{Shaw2012}. In fact, in only 17\% of cases was the operator or SCADA system listed as the initial identifier of the leak, while the public or emergency responders identified 30\% of leaks \citep[p. 3-39]{Shaw2012}. How do those spills occur, even with sophisticated computer programs in place? In-depth reports by the National Transportation Safety Board provide some insights...
	
	\bibliography{bibliography}
	
\end{document}