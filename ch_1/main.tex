\documentclass[12pt, man, natbib]{apa6}
\usepackage[USenglish]{babel}
\usepackage{setspace}
\usepackage{hyperref}

\title{All Quiet on the Pipeline Spill Front? When Learning Does Not Solve the Problem}
\shorttitle{All Quiet on the Pipeline Spill Front?}
\author{Julian Barg\\barg.julian@gmail.com}
\affiliation{Ivey Business School}
\setcitestyle{authoryear, open={()},close={)},citesep={,},aysep=}

\abstract{Pipeline spills are both frequent and serious pollution events. Since the 1990s, the industry has made great strides in developing computerized equipment to prevent and control leaks. Yet, pipeline spills today are almost as prevalent as they were in the 1980s, raising the question of why technology does not allow us to overcome the problem of normal accidents, even in simple systems. This paper juxtaposes learning in the pipeline industry on the incident and population level. The results indicate that broad, sweeping technological changes can miss the mark. Social processes, including the regulator that is supposed to act as a watchdog, can function to mask this problem by promoting the technological "solution".}

\begin{document}
	
	\maketitle
	
	\singlespacing
	
	\section{}
	
	Organizational learning comes down to choices. Firms can either invest in improving existing technology, or develop new technology \citep{March1991}. Investing in the "wrong" technology can lead to technological lock-ins \citep{Levinthal1993}. The actors in the pipeline industry have selected a number of technological solutions to resolve their most pressing issue. When a pipeline spill occurs, the oil quickly infiltrates the soil and seeps into the groundwater.\footnote{The infiltration depth in sand is assumed to be over 10m in the first day alone \citep{Bonvicini2015}.} The environmental degradation caused by oil affects the local environment, and the local populace, too: a 2019 sibling comparison study on oil spills in Nigeria found that in localities that are affected by oil spills, for every 1,000 live births, an additional 38.3 neonatal deaths occur\citep{Bruederle2019}. % Potentially add impact of spills on industry? Stigmatized industry.
	
	In their fight against pipeline spills, pipeline operators employ a variety of technologies, such as smart pigs, leak detection systems, and SCADA systems. Smart pigs, while traveling through the pipes, utilize electromagnetic flux or ultrasonic probing to assess corrosion or mechanical damages to the pipe \citep{Singh2017-7}. Internal leak detection systems measure the flow of oil at two points A and B to detect any loss in between those points. External leak detection systems detect signs of escaping hydrocarbons, and include acoustic, hydrocarbon, and temperature sensors, etc. \citep{Shaw2012}. Finally, SCADA systems are systems that allow an operator remotely monitor and operate lines. The operator typically sees on his screen charts of the flow at different points, can open and close valves, and startup or shutdown delivery of oil. Alarms from leak detection systems of the line are also displayed to the SCADA operator.\footnote{Larger pipeline companies operate control centers where all lines in a region are managed. Operators usually operate multiple SCADA systems at once, and more experienced employees supervise the operators. Control centers are operated in formal hierarchy, where for certain operations (such as clearing an alarm), a SCADA operator will require the go-ahead from a supervisor. See \citet{NTSB2012} for an in-depth description of an Enbridge control center in Edmonton as of 2012.}

	The high technology character of leak detection stands in contrast to the experienced reality of pipeline spills. A 2012 study commissioned by the Pipeline and Hazardous Materials Safety Administration (PHMSA) of onshore pipeline spills that occurred over a 19 month period, SCADA systems assisted in less than 25\% of cases with the detection and confirmation of the spill \citep[p. 3-33]{Shaw2012}. In only 17\% of cases was the operator or SCADA system listed as the initial identifier of the leak, while the public or emergency responders identified 30\% of leaks \citep[p. 3-39]{Shaw2012}. Why do the great learning efforts by pipeline operators fail to deliver the safety improvements that one would expect to see? A 2012 report prepared by the National Transportation Safety Board (NTSB) on the Kalamazoo River oil spill provides a good starting point for understanding the problem. A regional manager of Enbridge is quotes as saying: "...I'm not convinced [that there is a problem]. We haven't had any phone calls. I mean it's perfect weather out here--if it's a rupture someone's going to notice that, you know and smell it" \citep[p. 100]{NTSB2012}.
	
	Existing research has provided us with many insights on organizational learning. Learning is generally regarded as a function of experience, which builds knowledge, "embedded in a variety of repositories, including individuals, routines, and transactive memory systems" \citep[p. 1124]{Argote 2011}. For instance, an organization may notice that a routine fails under specific circumstances, and subsequently change its employee manual. An involved employees may also make a mental note of how to respond to the problem and whom to contact. Or an engineer might observe a method of speeding up production, and make a corresponding change to the production process. This knowledge-based approach to learning generally seeks to identify the factors that promote or hold back learning in an organization \citep[e.g.,][p. 2]{Argote2013-1}.
	
	In terms of repositories, routines, and transactive memory systems, the pipelines has made great strides over our qualitative observation period. But the quantitative data does not indicate corresponding outcomes in safety improvements. 
	
	
	
	
	
	In the context of pipeline safety, the [have obtained a lot of knowledge].\footnote{For instance, the original learning curve literature was very much hit-or-miss with regard to predicting learning\citep[pp. 321f]{Levitt1988}.}
	
%	Explain why that might be the case--complexity!

%	Past research has provided us with many insights on organizational learning. [Summarize Argote 2011, juxtapose with March missing feedback from environment]
	
	\bibliography{bibliography}
	
\end{document}