\documentclass{article}

\usepackage{natbib}
\usepackage{hyperref}
\usepackage{pifont}

\newcommand{\cmark}{\ding{51}}
\newcommand{\xmark}{\ding{55}}

\title{On Pipeline Spills as an Example of Managing Grand Challenges}
\author{Julian Barg}

\begin{document}

\maketitle

\citet{George2016} introduces a concise framework for the systematic study of societal grand challenges. This document lays out the fit between my research stream on pipeline spills and the editorial's framework for studying grand challenges.

\bigskip
\begin{small}
\centering
\begin{tabular}{l c c c}
	Dimension 					    & Empirical Context 	& Learning 	& Greenwashing 	\\
	\hline
	Articulating and Participating  & \cmark		   		& \xmark	& \xmark	\\
	Actor Needs and Aspirations		& \cmark				& \cmark	& \xmark	\\
	Societal Barriers				& \cmark				& \cmark	& \cmark	\\
	Organizational Constraints		& \cmark				& \cmark	& \xmark	\\
	Institutional Contexts			& \cmark				& \xmark	& \cmark	\\	
	Multilevel Actions				& \cmark				& \xmark	& \cmark	\\
	Coordinating Architectures		& \cmark				& \xmark	& \cmark	\\
	Reinforcing Mechanisms			& \cmark				& \cmark	& \cmark  	\\
	Outcomes and Impact				& \xmark				& \xmark	& \xmark	\\
	
\end{tabular}
	\end{small}

\subsection*{Articulating and Participating}

Pipeline spills are part of a larger process of local pollution in the United States, in the recent past most prominent in the form of plastic pollution, and "forever chemicals" (PFAS). Concerns exist especially in relationship to the grand challenge of ensuring the future supply of clean water. This risk of constructing pipelines across and near aquifers is articulated most prominently in North America by indigenous groups such as the \textit{water protectors}.\footnote{\url{https://www.nationalgeographic.com/news/2017/01/tribes-standing-rock-dakota-access-pipeline-advancement/}, accessed 2020-09-30.} Other than these groups however, only those directly affected by pipeline spill have have raised the issue. 

The context is a missed opportunity for research, as oil spills epitomize local environmental pollution, and pipeline safety is a tangible example of organizational performance beyond profit maximization. The data provided by the PHMSA showcases the problem, but has yet to be picket up by outsiders. Where industry level actors articulate the problem, it is unsurprisingly done with much more optimistic and positive undertones. The overall lack of interest in the topic is a factor in the continuance of the problem. The context offers some instances of articulating but has received little attention from scholarship \citep[the exception being][]{Estes2019}. Articulating and participating is not at the center of the research I am planning to do, but it is present in the context, could receive some marginal attention, and holds promise for future research in the stream.

A purpose of this research stream is to problematize the taken-for-grantedness of local environmental pollution events and the exogenous risks of complex systems. While \citet{George2016} intend for Articulating and Participating to mean that these are necessary factors for successful management of a grand challenge in the empirical context, in the case of pipeline, the drought of research and marginalization of voices contributes to a continuation of the challenge. Showcasing the data in empirical research itself would be an act of articulating that invites participation by other scholars.

%Successful work on grand challenges requires great mobilization. My research stream raises pipeline spills as a source of unintended, but tolerated pollution. Findings on pipeline spills could draw attention to other instances of salient pollution, and away from more abstract, aggregated metrics. Further, my intention is to make the raw data and analysis available to others, to enable subsequent research.

\subsection*{Actor Needs and Aspirations}

In the recent past, industry level actors have articulated the aspiration (at least in their rhetorics) to reduce pipeline spills to zero. At the same time, pipelines are promoted as a safe way of transporting oil, and the API and AOPL use more than questionable statistics to misrepresent pipeline as safe--implying that they do not see an urgent need to further improve pipeline safety.\footnote{\url{https://aopl.org/documents/en-us/d904059a-c130-41f9-b8da-3ca7e100ad4a/1} pp. 25ff, accessed 2020-10-01.} Aspirations are seen as an important mechanism for learning in the \textit{organizational routines} literature \citep{Baum2007}.

The qualitative data on pipeline spills includes the accounts of residents that are affected by pollution. Plenty of quantitative work exists on pollution, in particular on cases where the water supply is affected \citep{Schwarzenbach2010}. In the qualitative space on the other hand, literature is more likely to cover social issues, even when environmental pollution is involved \citep[e.g.,][]{Whiteman2016, Montgomery2019, Ferguson2005}. My current proposal does not include an analysis of the accounts of the affected, but it could be done in the future.

\subsection*{Societal Barriers}
\label{sec:soc}

Societal barriers are an element that is present in the context in different forms. One of the barriers, the technological barrier I touch on in the learning chapter. Further, the fossil fuel industry enjoys a wide support in the American population. This support constitutes a barrier to change. My work indirectly touches on this when I show in the greenwashing chapter that the industry fosters this support. 

%The greenwashing chapter shows that the industry targets the population with campaigns, and probably also through other channels, to garner their support. The greenwashing strategy of fostering a high tech image is directly concerned with creating this societal barrier to divestment by painting the industry as safe.

\subsection*{Organizational Constraints}

Organizational constraints are a common thread in the BTOF literature, as an extension of bounded rationality. My second empirical chapter on organizational learning also touches on constraints that are taken from the literature on complex systems. Organizational constraints are central to the work I will carry out, and the data communicates that a stagnation of pipeline safety is not too surprising. Similar developments are to be expected from other industries.

\subsection*{Institutional Contexts}
\label{sec:inst}

This research stream identifies institutional factors of a persistent status quo rather than seeking out individual variation at the periphery of the bell curve to brandish as examples of sustainability. A wide amount of factors fall into the umbrella of institutional context, including societal norms (see \nameref{sec:soc}), regulations, and the political system. In the context of pipeline spills, the institutional context is very complex, with multiple regulators and stakeholders. The greenwashing paper touches on this, but there is much left on the table for future research.

%There are two industry level actors, AOPL and and API. There are also multiple regulators, most importantly the PHMSA, NTSB, and DOT. A (maybe not so-) unique phenomenon at play is the regulatory capture of the PHMSA, which NTSB documents allude to. The oil industry also takes a special position in the American society and history \citep{Dochuk2019}, and enjoys broad political support, especially from republicans. The greenwashing phenomenon is related to many of these actors, although industry level actors and individual operators will stand at the center of the paper I plan on writing.

%The recent data does not show a general trend toward better pipeline safety. Therefore, rather than asking "which organizations are sustainable?" my work asks "why are we witnessing this sideways trend?". Barriers (not only societal ones) are central to this analysis. One obvious form of barrier is the regulatory capture, which to some degree extends to the general populace. This is to some degree covered by the greenwashing paper.

\subsection*{Multilevel Actions}

Pipeline spills allow us to zoom in to individual spills and even individual personnel, and zoom out to the level of population level actors. The coordination between industry level actors and individual operators, and their interaction with the regulator is a unique feature that is partly covered in my greenwashing chapter. Again, actions are seen as a reason for the persistence of the status quo, not for successful action on the grand challenge.

\subsection*{Coordinating Architectures}

The coordinating architecture in the context of pipeline spills are the industry level actors, and some of the captured regulators which reinforce the role of the API and AOPL--for instance, the PHMSA and NTSB repeatedly redistribute work to the API and AOPL. The learning paper tangentially touches on this architecture, while highlighting the role of industry level actors and networks within the industry are a central feature of the greenwashing paper.

\subsection*{Reinforcing Mechanisms}

My learning paper highlights one reinforcing mechanism \textit{for} the persistence of the problem, which is the complexity of the pipeline system that obscures the lack of progress. There are probably more reinforcing mechanisms at play in the empirical context, which I am yet to uncover. The \nameref{sec:inst} could probably be reframed to highlight its function as a reinforcing mechanism that preserves the status quo.

\subsection*{Outcomes and Impact}

Again, \citet{George2016} are probably referring to positive outcomes and impacts, which the comtext of pipeline spills does not offer. I content that to raise narratives of environmental pollution and the externalities of complex systems is also of importance.

\subsection*{Summary}

\citet{George2016} is a great framework for identifying the factors that have led to the current status quo in the pipeline industry. It allows us to identify many factors that lie outside the current scope of my work, and that could be picked up in future research. At the same time, the pipeline industry allows for an application of \citet{George2016} in a unique way, not to identify successful organizations, but to explain the current stagnation. To justify the relevance of this work, I need to show the parallels between this context and other empirical contexts (i.e., other industries).

The context of pipeline spills epitomizes failure on grand challenges. The challenge has persisted for a long time, and data indicates that pipeline spills will continue to plague operators and residents in the future. Much of the grand challenges literature focuses on identifying individual organizations that show promise for making progress on a challenge, less empirical work exists in management research on the systemic causes of grand challenges \citep[the exception being][]{Wright2017}.

The table shows that the two empirical chapters I plan to write cover many areas of the framework introduced by \citet{George2015}. Some opportunities are left on the table for future research, as laid out in the text.

\bibliography{../../bibliography}
\bibliographystyle{apalike}

\end{document}