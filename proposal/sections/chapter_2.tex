\section{Chapter 2: I know that they know nothing. Knowledge and organizational learning}

\begin{singlespace}
	\begin{quote}
		"Many organizational endeavors are more dependent on the sharing of understandings (reliability) than on their correctness (validity)" -- James G. March in \textit{The Ambiguities of Experience} \citep[p. 69]{March2010}.
	\end{quote}
\end{singlespace}

Organizational learning is among the oldest of topics discussed in management research. \citet{March1963} discusses learning extensively, especially in conjunction with politics. Empirical research on learning curves goes back to the 1930s \citep{Wright1936}. Today, our understanding of the topc has reached an astounding scope \citep{Argote2013}. There exist two main views of learning. (1) The \textit{organizational routines} view describes learning as organizations developing routine responses to reoccurring scenarios. (2) The \textit{organizational knowledge} view characterizes organizations as developing a knowledge and understanding of the world, which is stored in individuals, routines, or transactive memory systems \citep{Argote2011}.

The main difference between the two approaches to organizational learning is their take on knowledge. The \textit{organizational knowledge} stream describes knowledge in relatively tangible terms. Organizations create knowledge from their experience, it is held in various different forms, and can be transferred between business units. Still, organizational knowledge in this stream \textit{is} a metaphor: the stream distinguishes between organizational knowledge and knowledge in its literal sense, as held by individuals \citep{Argote2011}. Authors in the \textit{organizational routines} stream are hesitant to allude to the concept of knowledge (1) because they are aware that organizations are not monolithic, and diverging perspectives can exist within an organization. More importantly, the second group (2) questions the capability of organizations to accurately "know" something, in light of the sparse and uncertain signals that organizations receive from the environment \citep{March1975}. This chapter discusses the concepts of \textit{validity} and \textit{reliability} of knowledge in the context of the two separate conceptualizations of organizational learning.

The ideal outcome of a discussion on organizational knowledge would be to bridge the two existing streams. That "unification" of the two streams may be difficult to achieve, because of the fundamental disagreement regarding knowledge between the two streams. Instead, this chapter introduces two concepts that at least allow for a better understanding of the fundamental rift, and maybe a partial integration. "Reliablity" describes the degree to which a piece of knowledge is shared across the members of an organization (or at another level), and validity describes whether the knowledge will enable the organization to better understand, predict, and control problems, for example technological limitations or bottlenecks \cite{Rerup2020}. On the one hand, these concepts allow for a discussion of knowledge that is compatible with the \textit{organizational routines} view. On the other hand, these concepts translate two of the main concerns raised by the the \textit{Behavioral Theory of the Firm} into the language of \textit{organizational knowledge}. Reliability talks to the political dimension of organizational behavior, and validity talks to the ambiguous nature of experience. Hence, the concepts allow for a limited integration of ideas from the two stream.

This work is motivated by events outside of the literature. Organizational learning, and especially the \textit{organizational knowledge} stream of the literature, implicitly advances a strong modernistic view of the world. To say that learning creates knowledge and can be measured in performance improvements is ultimately a tribute to enlightenment. The \textit{organizational routines} stream deviates from this formula to some extend, as it traces back its roots to research on bounded rationality \citep{March1963}. But contributors to the organizational routines stream \textit{are} also seeking to enable organizations to identify and correct their shortcomings (such as inaccurate models of the world) \citep[e.g.,][]{Argyris1978}, and ultimately are motivated by a desire for progress. Not by accident did March discuss in his last work truth, justice, and beauty--Plato's transcendentals that later became the three tenants of enlightenment \citep{March2010}. While the critical interrogation of the concept of knowledge, and the inconsistencies that this review raises may be more in line with a postmodern inquiry, the more important contribution this review is to bring attention to organizational learning where it is based on invalid or unreliable knowledge. There may be limits to an organization's ability to accurately predict the future developments on the basis of past experience, but a lack of validity and reliability has clear implications that this review alludes to. Invalid knowledge will lead to strategic actions that are much less likely to accomplish their goals--think a society that makes political decisions based on fake news. Unreliable knowledge on the other hand will lead to learning and action that is not stable--for instance the implementation of a strategy that can easily be challenged and undone by adversaries. These are the two main point that this article draws attention to.

%This literature review focuses on organizational learning. The purpose of this chapter is to shed light on the intricacies of learning in the pipeline industry from a theoretical perspective. The first section summarizes the literature of the \textit{organizational knowledge} approach. In particular, the section emphasizes the strength of the knowledge-based approach, which accurately describes the accumulation of a knowledge stock that does take place in an organization when the goal is more or less clear and the environment stable. Further, This first section summarizes some of the other accomplishments of the literature that roughly fall into the knowledge-based stream, specifically its predecessor the learning curve literature, as well as vicarious learning, and population level learning.

%The second section summarizes the \textit{organizational routines} view of organizational learning. In particular, this section highlights the gaps left by the knowledge-based approach that may be filled by work in the behavioral space. The behavioral approach appreciates the nonlinearity and messiness of learning. Significant learning can results from individuals or groups organization that seek for the organization to break with convention \citep{Argyris1978}. This form of learning poses a challenge to the knowledge-based literature, because the changes that take place fall outside the normal rubric of improvement (as exemplified by learning curves). Other examples of these qualitatively different dimensions of learning in the literature are exploration and exploitation \citep{March1991}, and high intellect vs. low intellect learning \citep{March2010}.

%Building on the review of the two approaches to learning, this literature review then discusses two critical issues, which become particularly important in the context of pipeline spills. First, in light of the leveling off of the learning curve \citep[which in the pipeline industry takes the shape of a "baseline" of spills, or "normal accidents"][]{Perrow1984}, this section addresses the merits of less complex technologies. The learning literature is implicitly technocentric, but when efficiency does not take the primacy--as is the case when dealing with toxic chemical such as oil-- rather than adding and improving technologies, scaling back and relying on less complex technologies may be a better way to go.

%Second, in light of the damages that have been brought about by the petroleum industry, it is timely to take a critical look at the modernistic assumptions of the learning literature. The learning literature should confront the fact that new technolgies also routinely bring about new problems \citep{Beck1992}. To stay with the theme of this dissertation: pipeline spills (and climate change) are problems that did not exist before the petroleum industry came into being. Acknowledging the conjunction of technology and risks is not meant to take away from the recognition of its merits. Yet, we encourage a debate about the tendency of the literature to equate "newer" with "better" rather than "different".

% multiple learning curves either fall into a learning progress toward a more abstract goal, or in a less modernistic conceptualization, learning [verb] when the organization or population turns its head toward a new goal.