\subsection{Context and Data}

The empirical sections of this dissertation use data on the US pipeline industry. This empirical context offers a number of advantages over other datasets on environmental emissions and pollution. Compared to other industries, there is a good data coverage on pipeline spill from a diverse group of actors, including government agencies, the press, grassroot organizations, and industry level organizations. The documents on spills often include information on the events as well as causes. Further, the pipeline industry allows us to "zoom" in and out on individual oil spills and understand environmental pollution and impacts in a much more thorough fashion than is the case for other industries. Longitudinal quantitative data over a relatively long time frame is publicly available. Overall, pipeline spills as well as the subsequent learning process, and greenwashing attempts all play out in a public fashion.

Compared to environmental emissions and pollution by other industries, pipeline spills play out in a very public fashion. Pipeline spills are often initially discovered by members of the public \citep[p. 3-39]{Shaw2012}. Pipelines are built across the country, often on public land. Hence, large pipeline spills cannot rarely if ever be fully concealed from the public. Large oil spills also make up for a dominant share of the overall spill volume, hence even if some small spills are missed the data still offers an almost complete picture. When oil enters waterways, its color and smell makes the spill very obvious. Pipeline operators and employees are legally obligated to report pipeline spills.\footnote{See \url{https://www.phmsa.dot.gov/incident-reporting}, accessed 2020-08-30.} 

There are two main sources for data on pipeline spills in the US. (1) The Pipeline and Hazardous Materials Safety Administration (PHMSA) maintains a repository on all pipeline spills that occur. A fairly unique attribute of the data is that there is both qualitative and quantitative data available. Over 300 pipeline spills occur in the United States every year, and more than 100 of them are classified by the PHMSA as significant.\footnote{Meaning an injury, fire, explosion or property damage of over \$50,000, or the spill volume is at least 50bbls. See also \url{https://www.phmsa.dot.gov/sites/phmsa.dot.gov/files/docs/pdmpublic_incident_page_allrpt.pdf} and \url{https://julianbarg.shinyapps.io/incident_dashboard/}, accessed 2020-07-14.} The number of spills is sufficient for quantitative analysis to be sensible, but not at a level where an individual large spill becomes irrelevant. PHMSA also provides a dataset on pipeline operators, which allows for identification of the organization that caused each oil spills, how many miles of pipelines these organizational operate, and how much oil the organization transported over what distance each year.\footnote{See \url{https://github.com/julianbarg/oildata}, accessed 2020-07-14}. Between 2002 and 2004, PHMSA significantly increased the amount of data collected from each operator and on each spill. Hence, from 2004 forward the quality of the PHMSA data is generally good.

(2) The National Transportation Safety Board (NTSB) provides reports on pipeline spills that the agency deems significant. These reports typically have a length of 50-200 pages (usually being more than 100 pages long) and detail the incident, events leading up to it, and its causes. From 1969 until today, 142 reports and briefs have been published. The NTSB reports stand out from other reports because of the NTSB's "Go Team". Members of the Go Team are ready around the clock to be deployed to accident sites, for the benefit of collecting information that might otherwise not be available anymore.\footnote{See \url{https://www.ntsb.gov/investigations/process/Pages/default.aspx}, accessed 2020-08-31.} The NTSB also spends significant resources to determine underlying causes of (rather than liability for) the spills after the fact.\footnote{For instance, in one case NTSB tried to replicate an error of a SCADA system on a replica of the original SCADA setup \citep{NTSB2002}, and in another case NTSB used various pieces of heavy equipment on a pipeline section to determine what caused the mechanical damages that lead to a spill \citep{NTSB1990}.}

%PHMSA is the primary source for quantitative data, and NTSB the primary source for qualitative data. The NTSB reports are (obviously) biased toward very serious incidents, and the qualitative data available on less serious incidents is much less detailed. Only to some degree, this dearth of information can be counterbalanced by additional research on incidents that are not covered by NTSB, as some incidents are not reported on by any other source except for PHMSA. This lack of information on spills that are perceived as less impactful is a known limitation of research on pipeline spills. Both NTSB and PHMSA also have an overt focus on the direct impact of oil spills. The PHMSA dataset focuses on quantitative attributes of the spill, such as spill volume, volume of recovered oil, and boolean variables on remediation, whereas the NTSB focuses on the immediate impact, such as the magnitude of resulting fires, the number and types of injuries that occurred, and the immediate property damages caused. The PHMSA and NTSB data on the impact needs is supplemented with reports from residents, which both provide an understanding of the impact that the spills have on their lives, and provides a more tacit understanding of the impact on the local ecosystem. And of course the collection of third party data serves to triangulate information.

%There might either be a correlation between the \textit{complexity} of the incident and the \textit{information available} to us that is mediated by the \textit{severity} of the incident, or there could be just a correlation between the \textit{severity} of the incident and the \textit{information available}. In other words, we know whether severe incidents are more complex than less severe ones, but we do still know that they are complex.

%The qualitative and quantitative information provides us with three kinds of insights. In the first chapter, I carry out a panel regression to assess the state of organizational and population level learning in the pipeline industry. The qualitative data that is consulted indicates that specific issues are addressed through learning, but new, unique problems keep appearing, which prevents the industry from making further progress. The second chapter will not present any data, but rather discusses the learning literature while only sporadically touching on the phenomenon at hand. The third chapter uses discourse analysis and nonparticipant observation to explore the microfoundations of a coexistence of learning and a bottomed out learning curve.

In addition, other public agencies also sometimes become involved in th cleanup of pipeline spills and subsequently author reports. In addition to the quantitative data it provides, the PHMSA also occasionally commissions reports on pipeline safety \citep[e.g.,]{Shaw2012}. Other agencies that are involved with pipeline spills and provide relevant information are the Environmental Protection Agency (EPA), the National Oceanic and Atmospheric Administration (NOAA), Fire Marshals, and the coast guard. All these agencies sometimes provide primary data on the impacts of pipeline spills.\footnote{E.g., \url{https://response.restoration.noaa.gov/about/media/10-years-after-being-hit-hurricane-katrina-seeing-oiled-marsh-center-experiment-oil-clea}, accessed 2020-08-30}. Journalists also report on pipeline spills, and sometimes create very in-depth material that represents a valuable addition to government reports \citep[e.g.,][]{McGowan2012}. Finally, residents that are affected by pipeline spills sometimes organization in grassroot organizations. These grassroot organizations provide valuable data on the human impacts of pipeline spills.\footnote{E.g., \url{http://grangehallpress.com/Enbridgeblog/}, accessed 2020-08-30}.

% Paragraph on pipeline spills zoom in and out.

% Paragraph on quantitative data that is available.

%\subsection{Pipeline safety}