\section{Literatures}

Multiple literatures contribute to this dissertation, the obvious ones being \textit{organizational learning} and \textit{greenwashing}. The learning literature further falls into two camps: \textit{organizational knowledge} and \textit{organizational routines} \citep{Bingham2011}. Other literatures that inform this dissertation include the work on ESG indicators and \textit{grand challenges}. Most importantly, there is an emerging stream in management research that discusses the management of natural resources \citep{George2015}. This research stream's starting point is the observation that an excess emission of certain pollutants on a global scale would have catastrophic consequences \citep{Rockstrom2009a}, and the literature problematizes excess resource use.

%Multiple literatures contribute to this dissertation. The challenge will be to make these languages talk to one another. The literature on \textit{Grand Challenges} provides us with a research agenda, but has little to offer (so far) that can guide our work \citep{George2015}. In lieu of input from management research on the use of natural resources, we can draw on works in interdisciplinary journals that discuss use of \textit{natural resources} by organizations \citep[e.g., ][]{Rockstrom2009}. For theory development, this dissertation builds on the learning literature. The \textit{learning literature} again falls into two camps: (1) One stream traces back to the research on \textit{learning curves}. This stream looks to disaggregate organizational learning and identify the factors that accelerate or impede learning \citep{Argote2013}. (2) The other stream originates in the \textit{behavioral theory of the firm}. That stream looks at choices and systemic impediments to learning \citep[e.g., ][]{March1963, Levitt1988, Levinthal1993}. Beyond these literatures, \textit{pipeline technology} is discussed in engineering, a stream of sociology conducts \textit{disaster research}, and research on complex systems \citep[especially][]{Perrow1984} also contributes to this dissertation. These literatures will play a subordinate role.

\subsection{Grand challenges and ESG indicators}

The starting point for this work is the environmental part of the triple bottom line \citep{Elkington1997}. To just mention one in a long lineage of conceptualizations \citep{Bansal2017}: \citet{Rockstrom2009} argue that there are certain biogeochemical flows on earth, the condition of which constitutes a \textit{safe operating space}. Based on these, one can define quantitative \textit{planetary boundaries} for resource use. For instance, an ocean acidity above a certain value would lead to catastrophic results. This, and similar concrete environmental concerns have also entered management research \citep[e.g.,][]{Whiteman2013}. The social-ecological systems literature contends that to respond to these and other environmental concerns would require a rather unprecedented collective response \citep{Reyers2018}. 

In addressing to environmental concerns and the difficulty of responding to them, AMJ initiated a Special Research Forum to motivate more work under the banner of \textit{grand challenges} \citep{George2016}. Usually, the grand challenges literature showcases or theorizes how organizations can defy limitations and effectively tackle grand challenges against all odds \citep[e.g.,][]{Ferraro2015}. This dissertation turns the research on grand challenges on its head by taking as its starting point the status quo of pollution and improvements thereof--in a sense, the second order function of pollution. I pick an empirical context and investigate what emissions there are and what we can expect of the future. In that specific 

The "gold standard" for research on environmental sustainability is to comprehensively measure environmental impacts. A common approach for doing so is to use an \textit{ESG} indicator \citep{Montiel2014}. However, many barriers have to be overcome to make effective use of ESG indicators. Specifically, how the indicator is constructed has to be taken into consideration: the researcher has to be aware that the indicator is a product of social construction and has to treat it as such when conducting empirical research \citep{Eccles2019}. In particular, comparisons across industries are problematic. ESG indicators are always a combination of other metrics, and when for instance one of these metrics dominates the impact of an industry (e.g., downstream emissions of the fossil fuel industry), that should be taken into consideration during research design. Data availability also tends to be better for large corporations, favoring a cross-industry approach over intra-industry tests.

This dissertation sidesteps the issues associated with using an ESG indicator to some degree by using a more specific indicators that captures environmental pollution specifically. Moving from an ESG indicator to a more specific variable means making a sacrifice. The researcher at least to some degree forgoes the aspiration to measure impacts comprehensively, and research may become susceptible to greenwashing. For example, a chemical producer might try to improve its image by improving worker conditions; to then make any generalized statements on the sustainability of the corporation's operations without also taking into account e.g., environmental emissions would draw a wrong picture. On the flip side, to judge chemical company with excessive deaths only by its environmental impacts would also be flawed. By focusing on just one issue these complexities are lost.

The only context where focusing on just one metric would be justified is when that metric represents the most important area of impact. Coal power plants for instance are characterized by the high number of respiratory problems and indirect deaths they cause through air pollution, and the nuclear industry by its catastrophic potential. For pipelines, the case is less clear, because of their role in the fossil fuel supply chain, and by extension global climate change. However, pipelines are not indispensable for the global fossil fuel. A large share of petroleum transport globally happens by ship, which is also very cost-efficient. Thus, the pipeline industry's environmental impact is largely characterized by pipeline spills, especially because of their catastrophic potential.

\subsection{Organizational learning}

As mentioned above, the literature on organizational learning traces its roots back to two distinct streams of literature that can be distinguished by their definitions of learning. The first stream defines learning as "a change in the organization's knowledge that occurs as a function of experience" \citep[p. 1124]{Argote2011}. Henceforth, this stream will be called the knowledge-based approach. The second literature holds that organizations learn "by encoding inferences from history into routines that guide behavior" \citep[p. 320]{Levitt1988}. The contributors are much more careful about stating that organizations \textit{know} per se the lessons learned. Henceforth, this second stream of literature will be called the behavioral approach.

\subsubsection{Knowledge-based approach}

The first discussions of organizational learning are found in the learning curve literature \citep{Wright1936}. In particular, WW2 provided a couple of "quasi-experiments". In the shipbuilding industry, the researchers could observe how with every subsequent unit of production, productivity would improve \citep{Searle1945}. The most straightforward mathematical representation of the learning curve is the progress ratio. For instance, if the progress ratio is \textit{p}, then each time the cumulative output doubles, the unit cost would be predicted to drop to \textit{p}\% of its previous value \citep[p. 15]{Argote2013-1}. In other words, while in the beginning organizational learning allows for a quick reduction of unit cost, eventually, the next doubling of cumulative production is so far out that the unit cost is almost constant. One ambition of the learning curve literature is to mathematically disaggregate learning curves into multiple intraorganizational factors that predict the speed of learning \citep[e.g.,][]{Arrow1962}.

Because so many different factors were found to influence the learning rate, the literature eventually directed its attention to the process of organizational learning itself. A large share of this body of work roughly follows this pattern as exemplified by the structure of \citet{Love2014}: the author selects an organization-level performance variable (innovation, measured as sales from newly introduced products), and gathers this data for large companies in an industry or country (Ireland). Then, independent variables are selected that account for the heterogeneity across organizations (innovation linkages, measured as product development with customers or suppliers, joint ventures, etc.). This approach has allowed researchers to identify a broad variety of sources of variation \citep[pp. 18ff]{Argote2013-1}.

A limitation of this relatively formulaic approach however is that it may fail to identify path-breaking innovation. Not all knowledge is equally important, and the best insights are sometimes difficult to capture with quantitative metrics. For instance, some new pieces of knowledge might fall outside the regular schema of innovation and lead an organization into a new industry. And many fossil fuel companies are currently (still) successful because they double down on their existing knowledge stock and insulate their industry from changes--an orthodox learning paper might still diagnose learning, if, for example, production increases. But an example of a more interesting question with regard to learning would be which organizations manage to diversify and benefit from the rise of renewable energy.

%The weakness of this approach lies in the lack of its ability to compare different technologies and notice improvements that come about with multiple "generations" of technology or slightly different technologies that serve the same purpose. For instance, to go back to the origin of this literature, we might 

%knowledge transfer, vicarious learning, population level learning

\subsubsection{Behavioral approach}

The Carnegie school early on took notice of organizational learning \citep[e.g., ][]{March1963}. For some time, this literature developed in parallel to the learning curve literature. This difference between the two approaches is best exemplified by \citet{Argyris1978}. \citet{Argyris1978} developed the concept of double-loop learning. The first loop represents adjustments according to well-known decision criteria, such as launching a promotion when a sales goal is not met. The second loop represents an adjustment of the decision making process itself. For instance, a member of the organization may discover that the organization's goal has become unattainable, and push for a modification of the goal itself. This literature allows scholars to talk about issues that fall outside the scope of the learning curve (and knowledge-based) literature. A drawback is that it is difficult to translate the this literature into empirical work. For example, one major criticism of learning curves is that findings may have resulted from a self-fulfilling prophecy--an organization ends up at a certain productivity level \textit{because} that productivity level was the goal. The organization would not overaccomplish, because members lower their efforts when they approach the target. And if the organization falls short of its goal, the organization may move the goal post--by adjusting the goal, or its accounting approach.\footnote{Similarly, the reason why organiztional learning and learning curves appear to be omnipresent could be a result of a publication bias.} The behavioral approach provides us with a language to discuss these issue and similar issues. 

Concepts that speak to the phenomenon of pipeline spills include the aforementioned double-loop learning, exploration and exploitation \citep{March1991}, the competency trap \citep{Levitt1988}, and experiential learning under ambiguity \citep{March1975}. Some streams of the behavioral approach have cross-fertilized the knowledge-based stream. These include work on learning from rare events \citep{March1991b, Maslach2018}, and learning from failure \citep[e.g.,][]{Madsen2010}. These literature talk to some of the tensions that can be observed with regard to pipeline safety--an insistence on existing technology, and a lack of major overhauls, but also surges in activity in response to spills.

\subsection{Greenwashing}

Greenwashing describes a range of activities. This section provides three examples of greenwashing. This dissertation uses the third one, selective disclosure, which should be distinguished from the first two. The first, maybe the oldest one, describes an attempt of improving reputation through association with a signifier of ethics. For instance, "bluewashing" describes a corporation's effort to construe an association between a product and the United Nations \citep{Laufer2003}. 
Similarly, in marketing greenwashing describes falsely advertising a product as environmentally friendly \citep{Delmas2011}. For that purpose, a corporation does not necessarily need to make false claim. It may be sufficient to use green packaging and images of flowers. Many consumers can also be mislead with close-to-meaningless labels and certifications. 
Greenwashing has also long been a problem at the level of corporate governance \citep{Ramus2005}. With regard to environmental reporting, the term greenwashing as used in the literature often describes a process of selectively releasing positive information about one's environmental (or social) performance without also releasing negatives ones \citep{Lyon2011}. That approach of selective disclosure is insidious, because the information released are objectively true, yet, the picture of reality they paint is not accurate. The approach is also suitable for empirical research on organizations. \citet{Marquis2016} for instance operationalizes greenwashing as the difference between what share of metrics of environmental performance a corporation discloses and what share of total impacts these disclosed emissions make up for. For example, a corporation that discloses nine out of ten of its emission types, but where the missing emission type makes up for 90\% of its environmental impacts, would gain an abysmal score of $0.1 - 0.9 = -0.8$.

The abovementioned types of greenwashing have in common that none of the statements which organizations make are openly untrue, or even criminally liable. Note also that intent is not a necessary condition for greenwashing (although intent is often implied). For instance, \citet{Lyon2011} point out that a firm might itself be uncertain of its environmental impacts \citep[pp. 26f]{Lyon2011}. More universally, greenwashing can be defined as "any communication that misleads people into adopting overly positive beliefs about an organization's environmental performance" \citep[p. 225]{Lyon2015}. Within that definition, greenwashing takes on different forms. 
Since misleading communication constitutes greenwashing regardless of intent, a discussion of motivations and mechanisms for greenwashing is optional in empirical works on the topic. That definition of greenwashing as misleading communication regardless of intent describes well the developments taking place in the pipeline industry. Although the qualitative data provides several examples that suggest malicious intent, the divergence between asserted and observed pipeline safety to some degree resides at the industry level and is shared by all actors. The myth of pipelines as a safe technology is diffused at the population level by actors such as the American Petroleum Institute, the engineering profession, and shared technologies. The qualitative data allows us to make educated guesses as to the mechanisms at play at the different levels, but not to establish unambiguous intent and effect direction.

The divergence between communicated and observed pipeline safety amounts to more than just a decoupling process. The diffusion of pipeline technology ensures some coupling in the industry, and there is little evidence that technology is used throughout the industry in ways that it is not intended for. Decoupling does not equal greenwashing \citep{Lyon2015}. For example, if sustainability is decoupled in an organization because the sustainability does not have sufficient resources to effectively implement initiatives, and the organization then announces a major initiative, this communication would then qualify as greenwashing. However, not all decoupled activities result specifically in greenwashing, and not all cases of greenwashing are the result of decoupling--as mentioned above, greenwashing can also be a deliberate, malicious strategy. Decoupling could occur in other areas, such as R\&D activities, and greenwashing could result from other organizational processes, for example malice or misjudgment. This dissertation specifically discusses greenwashing that results from a technology that may function as designed, but does not deliver the results that it promises. Motives for greenwashing in this context vary. There are documents that are specifically written to testify that pipelines are safe to obtain permits \citep[e.g., discussed in][]{Stansbury2011}. But there are also grey areas, where financial interests and obligations to ensure pipeline safety are interweaved and cannot be disentangled. Decoupling certainly is not the encompassing or even dominant cause of cause of greenwashing in the pipeline industry.