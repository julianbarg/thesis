\section{Literatures}

% Multiple literatures contribute to this dissertation. The challenge will be to make these languages talk to one another. The literature on \textit{Grand Challenges} provides us with a research agenda, but has little to offer (so far) that can guide our work \citep{George2015}. In lieu of input from management research on the use of natural resources, we can draw on works in interdisciplinary journals that discuss use of \textit{natural resources} by organizations \citep[e.g., ][]{Rockstrom2009}. For theory development, this dissertation builds on the learning literature. The \textit{learning literature} again falls into two camps: (1) One stream traces back to the research on \textit{learning curves}. This stream looks to disaggregate organizational learning and identify the factors that accelerate or impede learning \citep{Argote2013}. (2) The other stream originates in the \textit{behavioral theory of the firm}. That stream looks at choices and systemic impediments to learning \citep[e.g., ][]{March1963, Levitt1988, Levinthal1993}. Beyond these literatures, \textit{pipeline technology} is discussed in engineering, a stream of sociology conducts \textit{disaster research}, and research on complex systems \citep[especially][]{Perrow1984} also contributes to this dissertation. These literatures will play a subordinate role.

\subsection{Greenwashing}

Greenwashing describes a range of activities. This section provides three examples of greenwashing. This dissertation uses the third one, selective disclosure, which should be distinguished from the first two. The first, maybe the oldest one, describes an attempt of improving reputation through association with a signifier of ethics. For instance, "bluewashing" describes a corporation's effort to construe an association between a product and the United Nations \citep{Laufer2003}. 
Similarly, in marketing greenwashing describes falsely advertising a product as environmentally friendly \citep{Delmas2011}. For that purpose, a corporation does not necessarily need to make false claim. It may be sufficient to use green packaging and images of flowers. Many consumers can also be mislead with close-to-meaningless labels and certifications. 
Greenwashing has also long been a problem at the level of corporate governance \citep{Ramus2005}. With regard to environmental reporting, the term greenwashing as used in the literature often describes a process of selectively releasing positive information about one's environmental (or social) performance without also releasing negatives ones \citep{Lyon2011}. That approach of selective disclosure is insidious, because the information released are objectively true, yet, the picture of reality they paint is not accurate. The approach is also suitable for empirical research on organizations. \citet{Marquis2016} for instance operationalizes greenwashing as the difference between what share of metrics of environmental performance a corporation discloses and what share of total impacts these disclosed emissions make up for. For example, a corporation that discloses nine out of ten of its emission types, but where the missing emission type makes up for 90\% of its environmental impacts, would gain an abysmal score of $0.1 - 0.9 = -0.8$.

The abovementioned types of greenwashing have in common that none of the statements which organizations make are openly untrue, or even criminally liable. Note also that intent is not a necessary condition for greenwashing (although intent is often implied). For instance, \citet{Lyon2011} point out that a firm might itself be uncertain of its environmental impacts \citep[pp. 26f]{Lyon2011}. More universally, greenwashing can be defined as "any communication that misleads people into adopting overly positive beliefs about an organization's environmental performance" \citep[p. 225]{Lyon2015}. Within that definition, greenwashing takes on different forms. 
Since misleading communication constitutes greenwashing regardless of intent, a discussion of motivations and mechanisms for greenwashing is optional in empirical works on the topic. That definition of greenwashing as misleading communication regardless of intent describes well the developments taking place in the pipeline industry. Although the qualitative data provides several examples that suggest malicious intent, the divergence between asserted and observed pipeline safety to some degree resides at the industry level and is shared by all actors. The myth of pipelines as a safe technology is diffused at the population level by actors such as the American Petroleum Institute, the engineering profession, and shared technologies. The qualitative data allows us to make educated guesses as to the mechanisms at play at the different levels, but not to establish unambiguous intent and effect direction.

The divergence between communicated and observed pipeline safety amounts to more than just a decoupling process. The diffusion of pipeline technology ensures some coupling in the industry, and there is little evidence that technology is used throughout the industry in ways that it is not intended for. Decoupling does not equal greenwashing \citep{Lyon2015}. For example, if sustainability is decoupled in an organization because the sustainability does not have sufficient resources to effectively implement initiatives, and the organization then announces a major initiative, this communication would then qualify as greenwashing. However, not all decoupled activities result specifically in greenwashing, and not all cases of greenwashing are the result of decoupling--as mentioned above, greenwashing can also be a deliberate, malicious strategy. Decoupling could occur in other areas, such as R\&D activities, and greenwashing could result from other organizational processes, for example malice or misjudgment. This dissertation specifically discusses greenwashing that results from a technology that may function as designed, but does not deliver the results that it promises. Motives for greenwashing in this context vary. There are documents that are specifically written to testify that pipelines are safe to obtain permits \citep[e.g., discussed in][]{Stansbury2011}. But there are also grey areas, where financial interests and obligations to ensure pipeline safety are interweaved and cannot be disentangled. Decoupling certainly is not the encompassing or even dominant cause of cause of greenwashing in the pipeline industry.