\section{Chapter 3: The Green Black Gold Blues. Diffusion of greenwashing in the pipeline industry}

\begin{quote}
	'[W]e are building a pipeline that is state of the art and will be the safest pipeline  ever build.' -- TransCanada President \& CEO Russ Gerling on the Keystone Pipeline\footnote{\url{https://youtu.be/ctx0H8XR51s?t=127}, accessed 2020-08-23.}
\end{quote}

\begin{quote}
	Although pipeline technology has improved, new pipelines are subject to proportionally higher stress as companies use this improved technology to maximize pumping rates through increases in operational pressures and temperatures, rather than to use this improved technology to enhance safety margins. -- Excerpt from a technical report that challenges the environmental impact assessment for the Keystone Pipeline \citep[p. 4]{Stansbury2011}
\end{quote}

For decades, activists have called out corporations for not walking the talk. In 1992, Greenpeace warned that the most powerful corporations' rhetorics mostly serve to distract the public, while these corporations fight off liability and accountability in the judiciary and legislative arena \citep{Bruno1992}. To describe this phenomenon of corporations building a false green image, grassroot movements have coined the term "greenwash". Greenwashing describes "any communication that misleads people into adopting overly positive beliefs about an organization's environmental performance" \citep[p. 225]{Lyon2015}. Generally, corporate communication with stakeholders is driven by their goals and interests. For instance, what corporations stay quiet about is as important as or more important than what they do disclose \citep{Kim2015}. Hence, what corporations say cannot be taken at face value. The greenwashing literature researches this problem with regard to environmental performance. The most common style that has been explored in the literature is disclosure of only positive information on environmental performance and withholding of bad news \citep{Lyon2011}. To date, research has focused on different types of greenwashing, their prevalence, and performance implications for corporations \citep{Marquis2016, Ramus2005, Seele2017, Kassinis2018}.

The existing research shows that greenwashing is a phenomenon that plays out not only at the level of individual organizations. In empirical research, to control for industry effects has become the norm \citep[e.g.,][]{Ramus2005, Marquis2016, Du2015, Testa2018}. The necessity to control for the industry indicates that there are important processes taking place within industries. Standards and research insights that are shared across an industry can act as templates for greenwashing, and organizations copy each other's greenwashing strategies. The greenwashing literature has yet to cover these processes. Under the watchful eye of stakeholders, entire industries such as mining, agriculture, or the energy sector have come under suspicion across the board and need to constantly put in efforts to legitimize their business models. In cases such as these, industry could even surpass organizational factors as a predictor for greenwashing. Hence, a discussion is overdue on the question: \textit{How does the industry affect organization's propensity to greenwash?}. Assisting with this question can other literature on inter-industry processes \citep[such as][]{Maguire2009, Hardy2020}.

To empirically demonstrate how industry-specific greenwashing strategies are diffused, this research turns to an industry where greenwashing has taken a very peculiar form. The pipeline industry uses the veil of engineering to present itself as safe, and pipeline technology as perfectly controllable, despite pipeline spills being a regular occurrence in the US. The public repository of the Pipeline and Hazardous Materials Safety Administration (PHMSA) holds data on both individual operators' pipeline miles and the volume of crude and refined petroleum transported. Further, the repository offers a description of and quantitative data on each minor and significant pipeline spills that has occurred in the US. The analysis of text data for this research relies on Natural Language Processing (NLP)--specifically, Topic Modeling--to determine spill causes and technology trends \citep{Hannigan2019}. The descriptions of individual spills reveal the shortcomings that individual operators exhibit in terms of pipeline safety. This data is matched with text data on pipeline safety strategy obtained from annual reports or, where available, safety reports. Annual or safety reports provide insight into the strategic plans and actions of operators. Next, data on headquarter location and executives' connections (BoardEx) surfaces networks within the industry. Finally, documents by industry-level actors such as the American Petroleum Institute (API) or the PHMSA unearth the latest industry trends. Greenwashing is given where non-substantive industry trends, rather than the operator's safety problems, determine individual operators strategic plans and action. By using operators' spill frequency and volume over time, we can ensure that effective measures are not accidentally flagged as non-substantive.

By researching greenwashing in the pipeline industry in the form of non-sustantive strategic plans and actions in the pipeline industry, this research expands the greenwashing literature. The empirical data reveals the flow of information within the industry, and the contribution of intra-industry networks to greenwashing. Greenwashing in the form of engineering and technology-centric communication also represents an addition to the literature. This form of greenwashing is particularly insidious, because an observer needs to first penetrate a layer of engineering and technology lingo, before the underlying issue can be surfaced. The addition of this form of greenwashing to the literature could help direct attention to other industries that have developed sophisticated forms of greenwashing which may be impossible for laymen to discern. Polluting industries that make intense use of new technologies are likely candidates to exhibit this form of greenwashing, for example chemistry, engineering, and construction. Exposing the role of industry-level actors such as the API, and an industry-wide propensity to greenwash also has relevance outside academic circles. Where the industry plays a role in greenwashing, policy makers and activists that seek to reign in greenwashing need to take a more systemic view, and target industry-level actors, or industries as a whole. On a related note, this unique cross-level research, which spans from the industry down to individual spills, also contributes to the literature on social-ecological systems \citep{Reyers2018}.\footnote{More recently, the need for cross-level research has also been voiced repeatedly during the ARCS Online Seminar Series and Ivey Sustainability Salon. For instance, at the Ivey Sustainability Saloon session on July 16, 2020, Tima Bansal to Nicholas Poggioli: "If the firm is at one level, one could argue that the eco-system is a different level in which many actors interact. And, arguably, Sustainable Development is a macro-level concept (system of actors)."}

%Many corporations engage in greenwashing. So what, one may say, there is always a competitor one can buy from. But not every industry has an incumbent that can be trusted to act responsible and whose products are widely available. Especially in concentrated industries, finding a "good egg" may be difficult. Greenwashing in these industries can become an issue that is pervasive, that is "part of the culture", and that is hard to eradicate. This research takes a look at one such industry where greenwashing is pervasive to understand how greenwashing spreads across this industry. From the pipeline industry, the this research generalizes to industry-wide diffusion and proliferation of greenwashing.
%
%The existing literature discusses multiple different styles of greenwashing. Their commonality is that they encompass "any communication that misleads people into adopting overly positive beliefs about an organization's environmental performance" \citep[p. 225]{Lyon2015}. Selective disclosure describes a process whereby a corporation discloses favorable information about its environmental performance, whilst withholding negative information, to attain a positive image. When implemented correctly and in the 
%absence of a harsh, reliable punishments for greenwashing the market may reward this behavior with a higher valuation \citep{Lyon2011}. Usually, it is assumed that organizations greenwash to meet stakeholder expectations, to give the impression of transparency, or simply out of opportunism \citep{Kim2015, Marquis2016}. The definition of greenwashing does not prescribe a specific mechanism that would have to be at play for an activity to qualify as greenwashing.
%
%Pipeline operators use often new and unproven technology to window dress their performance and gain permits for the construction of new pipelines. The operators advertise their use of technologies that are "safe"--that can be proven to fulfill their function in a lab setting or simulation--but that have not served to drive down the number or volume of oil spills since the turn of the millenium (see section "Introduction"). The reason for this shortcoming is that these technologies do not address the causes of spills. We will denominate these technologies in this research as not issue-oriented. The industry maintains a strong claim to safety. Pipelines, unlike oil-by-rail, are asserted to be safe, by merit of the physical principles that they operate on. Further, modern technology, as employed to improve pipeline safety, is asserted to be perfectly controllable. Typically, greenwashing in the pipeline industry falls into the realm of nonmarket strategy. This becomes most obvious when documents are specifically written to obtain a permit or attain another goal. But greenwashing is also present on corporate websites, annual reports, and publications by industry associations like the American Petroleum Institute. These documents emphasize safety initiatives, and gloss over recent spills, or emphasize spill response and remediation efforts. This myth of safe and controllable pipeline technology has also allowed the pipeline industry to attain broad public and political support in many states.\footnote{In Louisiana for example, anybody who protests on or near pipelines faces up to five years in prison, hard labor, and a \$1,000 fine. \url{https://www.propublica.org/article/how-louisiana-lawmakers-stop-residents-efforts-to-fight-big-oil-and-gas}, accessed 2020-08-19}
%
%This research takes a longitudinal approach to greenwashing and draws on data for the period form 2002 to 2019 from multiple datasets to examine greenwashing in the pipeline industry. (1) The Pipeline and Hazardous Materials Safety Administration's (PHMSA) dataset on pipeline spills and (2) on pipeline miles operated by individual operators and their utilization allows for the identification of pipeline spills and their causes. (3) Documents by the American Petroleum Institute (API) and other industry-level actors allow us to monitor the emergence of new, non-substantive technologies that fall into the realm of pipeline safety, but do not address common causes of pipeline spills. (4) Public documents by pipeline operators, such as annual reports or dedicated safety reports show the diffusion of these technologies and their use for greenwashing. (5) Data on executives in the pipeline industry, obtained from BoardEx, reveal the web of connections along which rhetorics are diffused in the industry. Finally, (6) fines levied on pipeline operators by the PHMSA, (7) post-spill statements by pipeline operators, that reveal their rational and perceived errors (albeit only ex ante), and (8) accident reports by the National Transportation Safety Board (NTSB) allow us to qualitatively assess the divergence between technological developments and spill causes.
%
%To demonstrate the degree of greenwashing in the pipeline industry, this research determines the degree to which pipeline operators' communication with the external environment is driven by industry-level trends rather than by the individual operators' recent spills. This empirical examination also gives us an understanding of how greenwashing can become commonplace in an industry. The discussion of technology that is not issue-oriented also draws attention to other forms of greenwashing that front technology but miss the issue at hand. Other example include fracking, carbon offset, and possibly carbon capture and storage. This research may also allow stakeholders and activists to better target more insidious forms of greenwashing.
%
%\subsection{Greenwashing in the pipeline industry}
%
%Greenwashing is quite common in the American pipeline industry, and the fossil fuel sector in general \citep{Kassinis2018}. The industry extensively uses the language of engineering in its rhetoric. Its communication with the outside environment is typically carried out by engineers (e.g., in feasibility studies or environmental impact analyses). American Petroleum Institute (API) engineers define standards and explores new technologies for pipeline operators. The API simultaneously lobbies against climate action. It is difficult for the "receiving side" of the communication to see through the greenwashing strategy of the pipeline industry, unless they also have specialized engineers at their disposal.\footnote{For a successful attempt, see e.g., \citet{Stansbury2011}.} The challenge is not that the claims made by engineers in the pipeline industry are factually incorrect, but that they do not address the issues that most frequently lead to pipeline spills (e.g., they involve not issue-oriented technologies). The language of engineering that documents are dressed in prevent most readers to get through to that issue
%
%By taking advantage of the language of engineering, pipeline operators gain access to a wide array of greenwashing strategies. The obvious is a stated commitment to pipeline safety, dressed in the language of engineering, but without substantiating the action to be taken. In other cases, greenwashing takes a more insidious approach, wherein the language of engineering and pipeline safety will conceal the primacy of business interests. For instance, the language and engineering can be used to misrepresent investments with a financial interests as primarily motivated by safety, or to obtain objectively verifiable data that cannot be refuted by stakeholders that are laymen. The following are two examples: (1) after a 2010 oil spill that polluted the Kalamazoo River in Michigan, Enbridge agreed to spend at least \$110 Million to improve pipeline safety.\footnote{\url{https://archive.epa.gov/epa/newsreleases/united-states-enbridge-reach-177-million-settlement-after-2010-oil-spills-michigan-and.html}, accessed 2020-08-19.} Enbridge successfully translated this settlement into an investment by using the money to increase its pipeline capacity and transport more oil along the same route. The residents that had been affected by the spill lost out against Enbridge a second time: once again they had their lifes disrupted, this type by construction work in their backyard, sometimes on a very short notice. Enbridge successfully circumvented the need for an environmental impact assessment by replacing the pipeline in short sections.\footnote{For brief summary of events, see \url{https://www.youtube.com/watch?v=IAR7z76KWj8}, accessed 2020-08-08.}
%
%(2) In 2006, in a confidential document that was later leaked, a consulting company on behalf of the TransCanada Corporation made two claims. If the Keystone Pipeline was to be constructed the operator could using the latest technology detect a large spill in as little as 9 minutes, and any spill over 50 barrels would only occur once every seven years \citep{Consulting2006}. This claim was not testable at the time but did convince the regulator to greenlight the construction of the pipeline. The pipeline began operation in 2010, and as of 2020 had experienced 5 spills of over 50 barrels,\footnote{See \url{http://boldnebraska.org/keystone-pipeline-spill-history/}, or \url{https://julianbarg.shinyapps.io/incident_dashboard/}, accessed 2020-08-08.} including one case where the spill continued for 20 minutes after the affected landowner who had discovered the spill had called in\footnote{\url{https://www.thedickinsonpress.com/business/energy-and-mining/4004561-5-years-after-spill-rancher-and-pipeline-junkie-still-has}, accessed 2020-08-08}. The problems of the Keystone Pipeline are symptomatic of spill detection technology: a sensitive system can detect small spills but will also produce many false positives. Thus, the real challenge is actually the far more complex one of managing the safety culture, since personnel can easily become desensitized by frequent false alarms and even safety drills.\footnote{See e.g., \url{https://www.ntsb.gov/investigations/AccidentReports/Reports/PAR1201.pdf}, p. 101.} For a discussion of the role of the consulting firm in misleading the regulator in that case, see \citet{Stansbury2011}.
%
%\section{Assessing greenwashing in the pipeline industry}
%
%In order to empirically assess greenwashing in the American pipeline industry, this research focuses on two kinds of greenwashing. The first is non-substantive promises made. These are commitments to pipeline safety without details on action, or without the organization following through. This version of greenwashing can be coded by hand, as long as comparable documents are available across time. The second aspect are technologies that are not issue-oriented. Referrals to these technologies, we can capture by using Natural Language Processing (NLP): either through keyword searches, or by utilizing topic modeling.
%
%Since the quality of the text data will be the key to obtaining meaningful results, constructing a good sample is of the essence. A list of the largest pipeline operators can be extrapolated from the Pipeline and Hazardous Materials Safety Administration (PHMSA) dataset. To obtain good quality text data, my sampling strategy focuses on the largest players in the industry, and completeness across time. Where annual reports or similar documents cannot be obtained from central sources such as Mergent Archives, the SEC, or \url{www.annualreports.com}, the documents are collected as much as possible by hand from corporate websites. For the empirical section of this research, the two different types of greenwashing (non-substantive and technology-based) are tracked across organizations and across time. We obtain evidence that teh communication does in fact constitute greenwashing by comparing the diffusion of technology in rhetorics with actual spill rates and volumes. The empirical test then shows that the communication with the external environment on pipeline safety is determined by networks of diffusion for greenwashing much more than by recent failure modes, by substantive safety issues that could be addressed. To show that mismatch between communication and issues, this research takes advantage of both non-substantive communication and of the different types of technologies that are applied but that are not issue-oriented. These types of greenwashing we can track across the company documents and industry-level documents, and compare that data to the network data available from BoardEx. The importance of industry-level actors comes even more into focus when we turn to instances of deregulation in Louisiana and Texas. These instances of deregulation should act as exogenous shocks for greenwashing. We can expect that following these exogenous shocks the share space given to technologies that are not issue-oriented or non-substantive communication should become even more widespread.