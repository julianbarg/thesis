\subsection{Structure}

The first chapter focuses on the empirical context, the stagnation of pipeline safety. The chapter introduces the notion of convergence in performance measures, as introduced by the \textit{learning curves} literature, which later developed into the \textit{organizational knowledge} view \citep{Argote2013-1}. This observation of convergence, which we also see in the pipeline industry, leads to the first research question, \textit{how does the convergence of performance measures take place?} One might intuitively assume that when a performance measures stay constant, no learning takes place. The qualitative data speaks to this assumption, and shows that in the pipeline industry, indeed, organizational learning still takes place. The quantitative data is then used to show that while problems are addressed with learning, new, unique problems constantly emerge. This is consistent with research on complex systems and externalities of modern technology \citep{Beck1992, Perrow1984}. Finally, the discussion section raises the alternative view, \textit{organizational routines}, and introduces the notion that a further reduction of pipeline spills requires a more radical rethinking of existing paradigms, including technology that is used. If the status quo remains as is, the research suggests that pipeline spills will continue, despite organizations learning from spills.

%begins with an orthodox view of organizational learning. Organizational learning is a useful frame for analyzing the technological side of pipeline safety, and why certain safety improvements are attained. Qualitative data reveals the learning processes taking place within the industry. The usefulness of an orthodox theory of organizational learning ends where we can observe that learning continues, but no more improvements in pipeline safety are achieved (Figure 1). The learning literature predicts this bottoming out \citep{Argote2013-1}, but does not address whether learning curves converge because learning stops, or for other reasons. This chapter examines the mechanisms behind safety improvements, the limits to learning, and the bottoming out of pipeline safety.
% Change: What matters is were trying to understand the mechanism based on empirical observations.

%The second chapter raises the issue of validity in organizational learning \citep{Rerup2020}. The current consensus is that as organizations accumulate experience from performing a task, their performance increases \citep{Argote2011}. But, as demonstrated above, one can observe an accumulation of experience with a corresponding change in cognition--a process of organizational learning--without the accompanying change in performance. Outside the literature stream on \textit{organizational knowledge} \citep{Bingham2011}, authors emphasize the ambiguity of organizational experience \citep{March2010}. This stream would contend that sometimes, to attain success, a substantial break with precedent is necessary. This chapter reunites these two disparate streams of the organizational learning literature.
% high and low intellect, second loop learning, exploration and exploitation, competency trap

The second chapter begins as a review of the organizational learning literature. The review is divided into two sections. The first section maps out the \textit{organizational knowledge} stream of the literature, including some fundamental work on \textit{learning curves}. The second section outlines the literature on \textit{organizational routines}. As a next step, the chapter turns to the fundamental difference between the two approaches. The organizational knowledge stream only recognizes learning when it occurs within the current structure--such as improvements in a specific metric--whereas the organizational routines literature recognizes radical departures as learning. These radical departures may compete with or invalidate previous insights, which represents a considerable departure from organizational knowledge stream. To reconcile the two views to some degree, the chapter finally introduces the concept of \textit{validity} and \textit{reliability} of knowledge \citep{Rerup2020}.

The third chapter turns to greenwashing this chapter returns the focus to the empirical context of the pipeline industry. After a brief introduction of greenwashing, the chapter lays out the problem at hand in the pipeline industry. Pipeline operators use new technologies in their rhetorics to assure stakeholders that pipelines are safe. This greenwashing strategy is widely shared in the industry, indicating that in this context industry level actors are also involved. The chapter uses discourse analysis to show the presence of greenwashing as a strategy across the two levels of organizations and the industry. An encompassing quantitative analysis which employs Topic Modeling to process text data \citep{Hannigan2019} then reveals how greenwashing spreads across the industry through intra-industrial networks. Finally, in the discussion section the chapter introduces the notion of technology as a vehicle for greenwashing. %Add research question?

%The third chapter discusses greenwashing as a reason why pipelines continue to fail. Despite March raising the issue of goals, coalitions, and politics in his early work \citep{March1963}, the topic is inexplicably absent from the current literature. Internal industry standards and company practice show that new insights are incorporated into practice. But the lived reality of incidents and spills is not what shapes communication with external stakeholders. Instead, in the classic greenwashing fashion, outside-facing documents are carefully crafted to convey an image of pipelines as safe and responsible, and catastrophic spills or near-spills not being the norm, but rare exceptions--the actors craft a public image \citep{Lyon2015}. For instance, to obtain a permit for the construction of a new pipeline, a pipeline operator has to establish that the pipeline is safe. Similarly, support from the public and state governments requires for pipelines to be perceived as safe. The third chapter discusses how technologies can be used in greenwashing attempts to give an industry a modern, and safe image.