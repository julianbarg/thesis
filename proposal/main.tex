
\documentclass[12pt, man, natbib]{apa6}
\usepackage[USenglish]{babel}
\usepackage{setspace}
\usepackage{hyperref}

\title{A Couple of Spills a Year, That's Normal? Learning and Greenwashing in the Pipeline Industry}
\shorttitle{Thesis Proposal -- Julian Barg}
\author{Julian Barg\\jbarg.phd@ivey.ca}
\affiliation{Ivey Business School}
\setcitestyle{authoryear, open={()},close={)},citesep={,},aysep=}
\abstract{
	From 2000 to 2020, the standardized spill volume of refined petroleum pipelines has stayed constant at about 15 bbl per billion barrel-miles transported. In contrast, from 1980 to 2000, standardized spill volume had about halved. This dissertation researches why pipeline operators in the US keep causing and getting away with pipeline spills. The dissertation uses two lenses, organizational learning and greenwashing. These lenses reveal why, despite continuous efforts by engineers, the safety record of the industry has stagnated. The learning literature suggests that it is commonplace for organizational learning to converge at a high level of performance, as observed in the pipeline industry. Greenwashing is a strategy for organizations to escape negative consequences for poor environmental performance.
	
	The first chapter reveals the mechanisms behind the convergence in organizational learning. The empirical section uses a dataset of 6,147 pipeline spills, and qualitative data on 10 significant pipeline spill. This research reveals that valid learning only occurs in response to the spills that an operator experiences. For a general theory of learning, the empirical findings suggests that learning converges when the organization or system that learns has developed a high degree of complexity. Because of this complexity, learning in response to triggers such as failures is not sufficient anymore for making aggregate improvements. Learning turns into a perpetual game of whack-a-mole.
	
	The second chapter takes an encompassing look at the learning literature and promotes a more universal, new theory on the validity and reliability of learning. When learning goes beyond incremental improvements and touches on fundamental assumptions, organizations or industries can break out of their trajectory. However, many learning outcomes that have been thought of as "breakthroughs" have not led to the promised revolutions in the market. Validity and reliability fill an important gap in the literature of learning: even when learning produces sensible and internally consistent insights, these insights are meaningless if they do not serve for the organization to better understand, predict, and control existing problems, limitations, or bottlenecks--that is, if the knowledge is not valid. And valid knowledge still fails to make an impact if it is not reliable, meaning not shared across the organizational members who are to implement the insights.
	
	Finally, the third chapter discusses how pipeline operators keep in check the backlash for the environmental pollution that they cause. Pipeline operators shield themselves from criticism using new technology. When an operator causes a spill, the operator can point to the latest development in the constant flow of new technology as a remedy for future spills. The third chapter uses the same data on pipeline spills and pipeline networks as the first chapter, and adds text data from operators and industry level actors. I then track empirically how these patterns of greenwashing are diffused in the pipeline industry. This analysis sheds light on the role of industry level actors (such as the American Petroleum Institute) in greenwashing.
}

\keywords{organizational learning, greenwashing, industry level, population level, pipeline spills}

\begin{document}
	
	\maketitle
	
%	\singlespacing
	
	\frame{
	\insertsection{Introduction}
}

\begin{frame}
	\frametitle{Throwback}
	\framesubtitle{Pipeline spills, greenwashing, organizational learning}
	\begin{figure}
		\centerline{
			\includegraphics[scale=0.28]{kalamazoo.jpg}}
		\caption{\url{https://insideclimatenews.org/news/03052018/enbridge-fined-tar-sands-oil-pipeline-inspections-kalamazoo-michigan-dilbit-spill}}
	\end{figure}
	\note{
		The most real example of environmental pollution that I can think of. Quite countable. That is not to say that there isn't complexity, but it is very tangible. I think nobody calls pipeline spills fake news. Environmental pollution is my motivation.
	}
\end{frame}

\begin{frame}
	\frametitle{Tension}
	Technology \& Social-Ecological Systems
	\begin{figure}
		\centerline{
			\includegraphics[scale=0.25]{scada.jpg}}
		\caption{\url{https://insideclimatenews.org/news/20120919/few-oil-pipeline-spills-detected-much-touted-technology}}
	\end{figure}
	\note{
		Great technology in pipelines. Not too complex, which is not to say ineffective. I can understand. You can understand. But at the same time, pipeline spills are still common place. How is that?
	}
\end{frame}

\begin{frame}
	\frametitle{Framework}
	Juxtapose communication and reality of pipeline safety:
	
	\begin{enumerate}
		\item <1-> Reliability: The pipeline industry communicates a complex shared understanding of pipeline safety.
		\item <2> Validity: Over the last 20 years, developments in pipeline safety does not address sources of pipeline spill.
	\end{enumerate}
	\vspace{0.1cm}
	\hrule
	\vspace{0.1cm}
	\citeauthor{Rerup2020} (forthcoming)
	\note{
		Juxtaposing two concepts that are a pair. Allows for talking about politics of pipeline safety through vehicle of greenwashing--reliability. And for looking into causes of pipeline spills, incident reports--validity.
		
		So mentally that is my structure, the idea that guides me through my dissertation. How far I will make that explicit though, I do not know yet.
	}
\end{frame}

\begin{frame}
	\frametitle{Phenomenon}
	\framesubtitle{Challenge 1: Communicate development in spill metrics}
	\includegraphics[scale=0.45]{population_learning_4.png}
	
	\note{
		We can see three things clearly. 
		
		\begin{enumerate}
			\item Crude pipelines show almost continuously improvements.
			\item Refined shows improvements into the early 2000s, then stays constant.
			\item For refined, it is hard to say whether there is one curve going on, or whether there is a decline, and then a standstill.
		\end{enumerate}
	
		Meaning that the reason for improvements in pipeline safety after 2000 is something that is specific to crude oil. Also, refined learning curve for post-2000 is indistinguishable from standstill. What has changed since the year 2000?
	}
\end{frame}

\begin{frame}
	\frametitle{Phenomenon}
	\framesubtitle{Challenge 1: Communicate development in spill metrics}
	\includegraphics[scale=0.30]{population_learning_6.png}
	\note{
		We can see three things clearly. 
		
		\begin{enumerate}
			\item Crude pipelines show almost continuously improvements.
			\item Refined shows improvements into the early 2000s, then stays constant.
			\item For refined, it is hard to say whether there is one curve going on, or whether there is a decline, and then a standstill.
		\end{enumerate}
		
		Meaning that the reason for improvements in pipeline safety after 2000 is something that is specific to crude oil. Also, refined learning curve for post-2000 is indistinguishable from standstill. What has changed since the year 2000?
	}
\end{frame}

\begin{frame}
	\frametitle{Phenomenon}
	\framesubtitle{Challenge 2: To what extend to introduce the context/qualitative data}
	\textbf{Example from AOPL media campaign}
	\begin{figure}
		\centerline{
			\includegraphics[scale=0.25]{tech2.png}}
		\caption{\url{https://aopl.org/305561/Page/Show?Slug=toolkit-technology}}
	\end{figure}	
	Video: \url{https://vimeo.com/366506379}
	
	\note{
		\tiny{		
		Give me your cynical response to that video. Mark, based on your experience with technology, something like TQM. My response was--this is probably the control room for training. How many operators really have these fancy new devices in mint conditions? How much more likely is it that there are devices that have not been upgraded for 20 years? Interestingly enough, even in this short excerpt we can see one of the things that the regulator would complain about. Who can really look at six screens at the same time? How many lines is this employee operating at the same time?
		
		This is where reliability comes into play. The industry very consistently communicates a set of technologies. We technology is comprehensible and approaches to problems are sensible. However, that the technology exists does not mean that it is used, or properly used. Reports such as NTSB on Kalamazoo show that there are many potential loopholes and error sources. The deeper I go into the qualitative data, the more I can show that. Guess why crude has improved and refined has not? Because coatings, that's my hypothesis why. Not the fancy technology, that is not necessarily broadly applied. Who knows how many operators really have these new decides?
		
		My best guess for the last slide as to why there are big changes for crude pipelines but not refined? Coatings! If the other technology is so great, it should also benefit refined pipelines.
		}
	}	
\end{frame}
	
	\subsection{Structure}

The first chapter focuses on the empirical context, the stagnation of pipeline safety. The chapter introduces the notion of convergence in performance measures, as introduced by the \textit{learning curves} literature, which later developed into the \textit{organizational knowledge} view \citep{Argote2013-1}. This observation of convergence, which we also see in the pipeline industry, leads to the first research question, \textit{how does the convergence of performance measures take place?} One might intuitively assume that when a performance measures stay constant, no learning takes place. The qualitative data speaks to this assumption, and shows that in the pipeline industry, indeed, organizational learning still takes place. The quantitative data is then used to show that while problems are addressed with learning, new, unique problems constantly emerge. This is consistent with research on complex systems and externalities of modern technology \citep{Beck1992, Perrow1984}. Finally, the discussion section raises the alternative view, \textit{organizational routines}, and introduces the notion that a further reduction of pipeline spills requires a more radical rethinking of existing paradigms, including technology that is used. If the status quo remains as is, the research suggests that pipeline spills will continue, despite organizations learning from spills.

%begins with an orthodox view of organizational learning. Organizational learning is a useful frame for analyzing the technological side of pipeline safety, and why certain safety improvements are attained. Qualitative data reveals the learning processes taking place within the industry. The usefulness of an orthodox theory of organizational learning ends where we can observe that learning continues, but no more improvements in pipeline safety are achieved (Figure 1). The learning literature predicts this bottoming out \citep{Argote2013-1}, but does not address whether learning curves converge because learning stops, or for other reasons. This chapter examines the mechanisms behind safety improvements, the limits to learning, and the bottoming out of pipeline safety.
% Change: What matters is were trying to understand the mechanism based on empirical observations.

%The second chapter raises the issue of validity in organizational learning \citep{Rerup2020}. The current consensus is that as organizations accumulate experience from performing a task, their performance increases \citep{Argote2011}. But, as demonstrated above, one can observe an accumulation of experience with a corresponding change in cognition--a process of organizational learning--without the accompanying change in performance. Outside the literature stream on \textit{organizational knowledge} \citep{Bingham2011}, authors emphasize the ambiguity of organizational experience \citep{March2010}. This stream would contend that sometimes, to attain success, a substantial break with precedent is necessary. This chapter reunites these two disparate streams of the organizational learning literature.
% high and low intellect, second loop learning, exploration and exploitation, competency trap

The second chapter begins as a review of the organizational learning literature. The review is divided into two sections. The first section maps out the \textit{organizational knowledge} stream of the literature, including some fundamental work on \textit{learning curves}. The second section outlines the literature on \textit{organizational routines}. As a next step, the chapter turns to the fundamental difference between the two approaches. The organizational knowledge stream only recognizes learning when it occurs within the current structure--such as improvements in a specific metric--whereas the organizational routines literature recognizes radical departures as learning. These radical departures may compete with or invalidate previous insights, which represents a considerable departure from organizational knowledge stream. To reconcile the two views to some degree, the chapter finally introduces the concept of \textit{validity} and \textit{reliability} of knowledge \citep{Rerup2020}.

The third chapter turns to greenwashing this chapter returns the focus to the empirical context of the pipeline industry. After a brief introduction of greenwashing, the chapter lays out the problem at hand in the pipeline industry. Pipeline operators use new technologies in their rhetorics to assure stakeholders that pipelines are safe. This greenwashing strategy is widely shared in the industry, indicating that in this context industry level actors are also involved. The chapter uses discourse analysis to show the presence of greenwashing as a strategy across the two levels of organizations and the industry. An encompassing quantitative analysis which employs Topic Modeling to process text data \citep{Hannigan2019} then reveals how greenwashing spreads across the industry through intra-industrial networks. Finally, in the discussion section the chapter introduces the notion of technology as a vehicle for greenwashing. %Add research question?

%The third chapter discusses greenwashing as a reason why pipelines continue to fail. Despite March raising the issue of goals, coalitions, and politics in his early work \citep{March1963}, the topic is inexplicably absent from the current literature. Internal industry standards and company practice show that new insights are incorporated into practice. But the lived reality of incidents and spills is not what shapes communication with external stakeholders. Instead, in the classic greenwashing fashion, outside-facing documents are carefully crafted to convey an image of pipelines as safe and responsible, and catastrophic spills or near-spills not being the norm, but rare exceptions--the actors craft a public image \citep{Lyon2015}. For instance, to obtain a permit for the construction of a new pipeline, a pipeline operator has to establish that the pipeline is safe. Similarly, support from the public and state governments requires for pipelines to be perceived as safe. The third chapter discusses how technologies can be used in greenwashing attempts to give an industry a modern, and safe image.

	\subsection{Context and Data}

The empirical sections of this dissertation use data on the US pipeline industry. This empirical context offers a number of advantages over other datasets on environmental emissions and pollution. Compared to other industries, there is a good data coverage on pipeline spill from a diverse group of actors, including government agencies, the press, grassroot organizations, and industry level organizations. The documents on spills often include information on the events as well as causes. Further, the pipeline industry allows us to "zoom" in and out on individual oil spills and understand environmental pollution and impacts in a much more thorough fashion than is the case for other industries. Longitudinal quantitative data over a relatively long time frame is publicly available. Overall, pipeline spills as well as the subsequent learning process, and greenwashing attempts all play out in a public fashion.

Compared to environmental emissions and pollution by other industries, pipeline spills play out in a very public fashion. Pipeline spills are often initially discovered by members of the public \citep[p. 3-39]{Shaw2012}. Pipelines are built across the country, often on public land. Hence, large pipeline spills cannot rarely if ever be fully concealed from the public. Large oil spills also make up for a dominant share of the overall spill volume, hence even if some small spills are missed the data still offers an almost complete picture. When oil enters waterways, its color and smell makes the spill very obvious. Pipeline operators and employees are legally obligated to report pipeline spills.\footnote{See \url{https://www.phmsa.dot.gov/incident-reporting}, accessed 2020-08-30.} 

There are two main sources for data on pipeline spills in the US. (1) The Pipeline and Hazardous Materials Safety Administration (PHMSA) maintains a repository on all pipeline spills that occur. A fairly unique attribute of the data is that there is both qualitative and quantitative data available. Over 300 pipeline spills occur in the United States every year, and more than 100 of them are classified by the PHMSA as significant.\footnote{Meaning an injury, fire, explosion or property damage of over \$50,000, or the spill volume is at least 50bbls. See also \url{https://www.phmsa.dot.gov/sites/phmsa.dot.gov/files/docs/pdmpublic_incident_page_allrpt.pdf} and \url{https://julianbarg.shinyapps.io/incident_dashboard/}, accessed 2020-07-14.} The number of spills is sufficient for quantitative analysis to be sensible, but not at a level where an individual large spill becomes irrelevant. PHMSA also provides a dataset on pipeline operators, which allows for identification of the organization that caused each oil spills, how many miles of pipelines these organizational operate, and how much oil the organization transported over what distance each year.\footnote{See \url{https://github.com/julianbarg/oildata}, accessed 2020-07-14}. Between 2002 and 2004, PHMSA significantly increased the amount of data collected from each operator and on each spill. Hence, from 2004 forward the quality of the PHMSA data is generally good.

(2) The National Transportation Safety Board (NTSB) provides reports on pipeline spills that the agency deems significant. These reports typically have a length of 50-200 pages (usually being more than 100 pages long) and detail the incident, events leading up to it, and its causes. From 1969 until today, 142 reports and briefs have been published. The NTSB reports stand out from other reports because of the NTSB's "Go Team". Members of the Go Team are ready around the clock to be deployed to accident sites, for the benefit of collecting information that might otherwise not be available anymore.\footnote{See \url{https://www.ntsb.gov/investigations/process/Pages/default.aspx}, accessed 2020-08-31.} The NTSB also spends significant resources to determine underlying causes of (rather than liability for) the spills after the fact.\footnote{For instance, in one case NTSB tried to replicate an error of a SCADA system on a replica of the original SCADA setup \citep{NTSB2002}, and in another case NTSB used various pieces of heavy equipment on a pipeline section to determine what caused the mechanical damages that lead to a spill \citep{NTSB1990}.}

%PHMSA is the primary source for quantitative data, and NTSB the primary source for qualitative data. The NTSB reports are (obviously) biased toward very serious incidents, and the qualitative data available on less serious incidents is much less detailed. Only to some degree, this dearth of information can be counterbalanced by additional research on incidents that are not covered by NTSB, as some incidents are not reported on by any other source except for PHMSA. This lack of information on spills that are perceived as less impactful is a known limitation of research on pipeline spills. Both NTSB and PHMSA also have an overt focus on the direct impact of oil spills. The PHMSA dataset focuses on quantitative attributes of the spill, such as spill volume, volume of recovered oil, and boolean variables on remediation, whereas the NTSB focuses on the immediate impact, such as the magnitude of resulting fires, the number and types of injuries that occurred, and the immediate property damages caused. The PHMSA and NTSB data on the impact needs is supplemented with reports from residents, which both provide an understanding of the impact that the spills have on their lives, and provides a more tacit understanding of the impact on the local ecosystem. And of course the collection of third party data serves to triangulate information.

%There might either be a correlation between the \textit{complexity} of the incident and the \textit{information available} to us that is mediated by the \textit{severity} of the incident, or there could be just a correlation between the \textit{severity} of the incident and the \textit{information available}. In other words, we know whether severe incidents are more complex than less severe ones, but we do still know that they are complex.

%The qualitative and quantitative information provides us with three kinds of insights. In the first chapter, I carry out a panel regression to assess the state of organizational and population level learning in the pipeline industry. The qualitative data that is consulted indicates that specific issues are addressed through learning, but new, unique problems keep appearing, which prevents the industry from making further progress. The second chapter will not present any data, but rather discusses the learning literature while only sporadically touching on the phenomenon at hand. The third chapter uses discourse analysis and nonparticipant observation to explore the microfoundations of a coexistence of learning and a bottomed out learning curve.

In addition, other public agencies also sometimes become involved in th cleanup of pipeline spills and subsequently author reports. In addition to the quantitative data it provides, the PHMSA also occasionally commissions reports on pipeline safety \citep[e.g.,]{Shaw2012}. Other agencies that are involved with pipeline spills and provide relevant information are the Environmental Protection Agency (EPA), the National Oceanic and Atmospheric Administration (NOAA), Fire Marshals, and the coast guard. All these agencies sometimes provide primary data on the impacts of pipeline spills.\footnote{E.g., \url{https://response.restoration.noaa.gov/about/media/10-years-after-being-hit-hurricane-katrina-seeing-oiled-marsh-center-experiment-oil-clea}, accessed 2020-08-30}. Journalists also report on pipeline spills, and sometimes create very in-depth material that represents a valuable addition to government reports \citep[e.g.,][]{McGowan2012}. Finally, residents that are affected by pipeline spills sometimes organization in grassroot organizations. These grassroot organizations provide valuable data on the human impacts of pipeline spills.\footnote{E.g., \url{http://grangehallpress.com/Enbridgeblog/}, accessed 2020-08-30}.

% Paragraph on pipeline spills zoom in and out.

% Paragraph on quantitative data that is available.

%\subsection{Pipeline safety}

	\subsection{Literatures}

Multiple literatures contribute to this dissertation, the obvious ones being \textit{organizational learning} and \textit{greenwashing}. The learning literature further falls into two camps: \textit{organizational knowledge} and \textit{organizational routines} \citep{Bingham2011}. Other literatures that inform this dissertation include the work on ESG indicators and \textit{grand challenges}. Most importantly, there is an emerging stream in management research that discusses the management of natural resources \citep{George2015}. This research stream's starting point is the observation that an excess emission of certain pollutants on a global scale would have catastrophic consequences \citep{Rockstrom2009a}, and the literature problematizes excess resource use.

%Multiple literatures contribute to this dissertation. The challenge will be to make these languages talk to one another. The literature on \textit{Grand Challenges} provides us with a research agenda, but has little to offer (so far) that can guide our work \citep{George2015}. In lieu of input from management research on the use of natural resources, we can draw on works in interdisciplinary journals that discuss use of \textit{natural resources} by organizations \citep[e.g., ][]{Rockstrom2009}. For theory development, this dissertation builds on the learning literature. The \textit{learning literature} again falls into two camps: (1) One stream traces back to the research on \textit{learning curves}. This stream looks to disaggregate organizational learning and identify the factors that accelerate or impede learning \citep{Argote2013}. (2) The other stream originates in the \textit{behavioral theory of the firm}. That stream looks at choices and systemic impediments to learning \citep[e.g., ][]{March1963, Levitt1988, Levinthal1993}. Beyond these literatures, \textit{pipeline technology} is discussed in engineering, a stream of sociology conducts \textit{disaster research}, and research on complex systems \citep[especially][]{Perrow1984} also contributes to this dissertation. These literatures will play a subordinate role.

\subsubsection{Grand challenges and ESG indicators}

The starting point for this work is the environmental part of the triple bottom line \citep{Elkington1997}. To just mention one in a long lineage of conceptualizations \citep{Bansal2017}: \citet{Rockstrom2009} argue that there are certain biogeochemical flows on earth, the condition of which constitutes a \textit{safe operating space}. Based on these, one can define quantitative \textit{planetary boundaries} for resource use. For instance, an ocean acidity above a certain value would lead to catastrophic results. This, and similar concrete environmental concerns have also entered management research \citep[e.g.,][]{Whiteman2013}. The social-ecological systems literature contends that to respond to these and other environmental concerns would require a rather unprecedented collective response \citep{Reyers2018}. 

In addressing to environmental concerns and the difficulty of responding to them, AMJ initiated a Special Research Forum to motivate more work under the banner of \textit{grand challenges} \citep{George2016}. Usually, the grand challenges literature showcases or theorizes how organizations can defy limitations and effectively tackle grand challenges against all odds \citep[e.g.,][]{Ferraro2015}. This dissertation turns the research on grand challenges on its head by taking as its starting point the status quo of pollution and improvements thereof--in a sense, the second order function of pollution. I pick an empirical context and investigate what emissions there are and what we can expect of the future. In that specific 

The "gold standard" for research on environmental sustainability is to comprehensively measure environmental impacts. A common approach for doing so is to use an \textit{ESG} indicator \citep{Montiel2014}. However, many barriers have to be overcome to make effective use of ESG indicators. Specifically, how the indicator is constructed has to be taken into consideration: the researcher has to be aware that the indicator is a product of social construction and has to treat it as such when conducting empirical research \citep{Eccles2019}. In particular, comparisons across industries are problematic. ESG indicators are always a combination of other metrics, and when for instance one of these metrics dominates the impact of an industry (e.g., downstream emissions of the fossil fuel industry), that should be taken into consideration during research design. Data availability also tends to be better for large corporations, favoring a cross-industry approach over intra-industry tests.

This dissertation sidesteps the issues associated with using an ESG indicator to some degree by using a more specific indicators that captures environmental pollution specifically. Moving from an ESG indicator to a more specific variable means making a sacrifice. The researcher at least to some degree forgoes the aspiration to measure impacts comprehensively, and research may become susceptible to greenwashing. For example, a chemical producer might try to improve its image by improving worker conditions; to then make any generalized statements on the sustainability of the corporation's operations without also taking into account e.g., environmental emissions would draw a wrong picture. On the flip side, to judge chemical company with excessive deaths only by its environmental impacts would also be flawed. By focusing on just one issue these complexities are lost.

The only context where focusing on just one metric would be justified is when that metric represents the most important area of impact. Coal power plants for instance are characterized by the high number of respiratory problems and indirect deaths they cause through air pollution, and the nuclear industry by its catastrophic potential. For pipelines, the case is less clear, because of their role in the fossil fuel supply chain, and by extension global climate change. However, pipelines are not indispensable for the global fossil fuel. A large share of petroleum transport globally happens by ship, which is also very cost-efficient. Thus, the pipeline industry's environmental impact is largely characterized by pipeline spills, especially because of their catastrophic potential.

\subsubsection{Organizational learning}

As mentioned above, the literature on organizational learning traces its roots back to two distinct streams of literature that can be distinguished by their definitions of learning. The first stream defines learning as "a change in the organization's knowledge that occurs as a function of experience" \citep[p. 1124]{Argote2011}. Henceforth, this stream will be called the knowledge-based approach. The second literature holds that organizations learn "by encoding inferences from history into routines that guide behavior" \citep[p. 320]{Levitt1988}. The contributors are much more careful about stating that organizations \textit{know} per se the lessons learned. Henceforth, this second stream of literature will be called the behavioral approach.

\textbf{Knowledge-based approach}

The first discussions of organizational learning are found in the learning curve literature \citep{Wright1936}. In particular, WW2 provided a couple of "quasi-experiments". In the shipbuilding industry, the researchers could observe how with every subsequent unit of production, productivity would improve \citep{Searle1945}. The most straightforward mathematical representation of the learning curve is the progress ratio. For instance, if the progress ratio is \textit{p}, then each time the cumulative output doubles, the unit cost would be predicted to drop to \textit{p}\% of its previous value \citep[p. 15]{Argote2013-1}. In other words, while in the beginning organizational learning allows for a quick reduction of unit cost, eventually, the next doubling of cumulative production is so far out that the unit cost is almost constant. One ambition of the learning curve literature is to mathematically disaggregate learning curves into multiple intraorganizational factors that predict the speed of learning \citep[e.g.,][]{Arrow1962}.

Because so many different factors were found to influence the learning rate, the literature eventually directed its attention to the process of organizational learning itself. A large share of this body of work roughly follows this pattern as exemplified by the structure of \citet{Love2014}: the author selects an organization-level performance variable (innovation, measured as sales from newly introduced products), and gathers this data for large companies in an industry or country (Ireland). Then, independent variables are selected that account for the heterogeneity across organizations (innovation linkages, measured as product development with customers or suppliers, joint ventures, etc.). This approach has allowed researchers to identify a broad variety of sources of variation \citep[pp. 18ff]{Argote2013-1}.

A limitation of this relatively formulaic approach however is that it may fail to identify path-breaking innovation. Not all knowledge is equally important, and the best insights are sometimes difficult to capture with quantitative metrics. For instance, some new pieces of knowledge might fall outside the regular schema of innovation and lead an organization into a new industry. And many fossil fuel companies are currently (still) successful because they double down on their existing knowledge stock and insulate their industry from changes--an orthodox learning paper might still diagnose learning, if, for example, production increases. But an example of a more interesting question with regard to learning would be which organizations manage to diversify and benefit from the rise of renewable energy.

%The weakness of this approach lies in the lack of its ability to compare different technologies and notice improvements that come about with multiple "generations" of technology or slightly different technologies that serve the same purpose. For instance, to go back to the origin of this literature, we might 

%knowledge transfer, vicarious learning, population level learning

\textbf{Behavioral approach}

The Carnegie school early on took notice of organizational learning \citep[e.g., ][]{March1963}. For some time, this literature developed in parallel to the learning curve literature. This difference between the two approaches is best exemplified by \citet{Argyris1978}. \citet{Argyris1978} developed the concept of double-loop learning. The first loop represents adjustments according to well-known decision criteria, such as launching a promotion when a sales goal is not met. The second loop represents an adjustment of the decision making process itself. For instance, a member of the organization may discover that the organization's goal has become unattainable, and push for a modification of the goal itself. This literature allows scholars to talk about issues that fall outside the scope of the learning curve (and knowledge-based) literature. A drawback is that it is difficult to translate the this literature into empirical work. For example, one major criticism of learning curves is that findings may have resulted from a self-fulfilling prophecy--an organization ends up at a certain productivity level \textit{because} that productivity level was the goal. The organization would not overaccomplish, because members lower their efforts when they approach the target. And if the organization falls short of its goal, the organization may move the goal post--by adjusting the goal, or its accounting approach.\footnote{Similarly, the reason why organiztional learning and learning curves appear to be omnipresent could be a result of a publication bias.} The behavioral approach provides us with a language to discuss these issue and similar issues. 

Concepts that speak to the phenomenon of pipeline spills include the aforementioned double-loop learning, exploration and exploitation \citep{March1991}, the competency trap \citep{Levitt1988}, and experiential learning under ambiguity \citep{March1975}. Some streams of the behavioral approach have cross-fertilized the knowledge-based stream. These include work on learning from rare events \citep{March1991b, Maslach2018}, and learning from failure \citep[e.g.,][]{Madsen2010}. These literature talk to some of the tensions that can be observed with regard to pipeline safety--an insistence on existing technology, and a lack of major overhauls, but also surges in activity in response to spills.

\subsubsection{Greenwashing}

Greenwashing describes a range of activities. This section provides three examples of greenwashing. This dissertation uses the third one, selective disclosure, which should be distinguished from the first two. The first, maybe the oldest one, describes an attempt of improving reputation through association with a signifier of ethics. For instance, "bluewashing" describes a corporation's effort to construe an association between a product and the United Nations \citep{Laufer2003}. 
Similarly, in marketing greenwashing describes falsely advertising a product as environmentally friendly \citep{Delmas2011}. For that purpose, a corporation does not necessarily need to make false claim. It may be sufficient to use green packaging and images of flowers. Many consumers can also be mislead with close-to-meaningless labels and certifications. 
Greenwashing has also long been a problem at the level of corporate governance \citep{Ramus2005}. With regard to environmental reporting, the term greenwashing as used in the literature often describes a process of selectively releasing positive information about one's environmental (or social) performance without also releasing negatives ones \citep{Lyon2011}. That approach of selective disclosure is insidious, because the information released are objectively true, yet, the picture of reality they paint is not accurate. The approach is also suitable for empirical research on organizations. \citet{Marquis2016} for instance operationalizes greenwashing as the difference between what share of metrics of environmental performance a corporation discloses and what share of total impacts these disclosed emissions make up for. For example, a corporation that discloses nine out of ten of its emission types, but where the missing emission type makes up for 90\% of its environmental impacts, would gain an abysmal score of $0.1 - 0.9 = -0.8$.

The abovementioned types of greenwashing have in common that none of the statements which organizations make are openly untrue, or even criminally liable. Note also that intent is not a necessary condition for greenwashing (although intent is often implied). For instance, \citet{Lyon2011} point out that a firm might itself be uncertain of its environmental impacts \citep[pp. 26f]{Lyon2011}. More universally, greenwashing can be defined as "any communication that misleads people into adopting overly positive beliefs about an organization's environmental performance" \citep[p. 225]{Lyon2015}. Within that definition, greenwashing takes on different forms. 
Since misleading communication constitutes greenwashing regardless of intent, a discussion of motivations and mechanisms for greenwashing is optional in empirical works on the topic. That definition of greenwashing as misleading communication regardless of intent describes well the developments taking place in the pipeline industry. Although the qualitative data provides several examples that suggest malicious intent, the divergence between asserted and observed pipeline safety to some degree resides at the industry level and is shared by all actors. The myth of pipelines as a safe technology is diffused at the population level by actors such as the American Petroleum Institute, the engineering profession, and shared technologies. The qualitative data allows us to make educated guesses as to the mechanisms at play at the different levels, but not to establish unambiguous intent and effect direction.

The divergence between communicated and observed pipeline safety amounts to more than just a decoupling process. The diffusion of pipeline technology ensures some coupling in the industry, and there is little evidence that technology is used throughout the industry in ways that it is not intended for. Decoupling does not equal greenwashing \citep{Lyon2015}. For example, if sustainability is decoupled in an organization because the sustainability does not have sufficient resources to effectively implement initiatives, and the organization then announces a major initiative, this communication would then qualify as greenwashing. However, not all decoupled activities result specifically in greenwashing, and not all cases of greenwashing are the result of decoupling--as mentioned above, greenwashing can also be a deliberate, malicious strategy. Decoupling could occur in other areas, such as R\&D activities, and greenwashing could result from other organizational processes, for example malice or misjudgment. This dissertation specifically discusses greenwashing that results from a technology that may function as designed, but does not deliver the results that it promises. Motives for greenwashing in this context vary. There are documents that are specifically written to testify that pipelines are safe to obtain permits \citep[e.g., discussed in][]{Stansbury2011}. But there are also grey areas, where financial interests and obligations to ensure pipeline safety are interweaved and cannot be disentangled. Decoupling certainly is not the encompassing or even dominant cause of cause of greenwashing in the pipeline industry.
	
%	\subsubsection{Chapter 1: stuff}

In chapter 1, I do amazing stuff.

	\section{Chapter 1: When the next spill is only a matter of time. Learning in the pipeline industry.}

\begin{singlespace}
	\begin{quote}
		"...I'm not convinced [that there is a problem]. We haven't had any phone calls. I mean it's perfect weather out here--if it's a rupture someone's going to notice that, you know and smell it" -- quote from a regional manager during the costliest onshore pipeline spill in history \citet[p. 100]{NTSB2012}.
	\end{quote}
\end{singlespace}

From 1980 to 2000, the US pipeline industry was a on a good track toward making pipelines safe. The standardized spill volume of refined petroleum pipelines more than halved, to a value of about 15 bbl per billion barrel-miles transported. After the year 2000, the development of new pipeline safety technology did not stop--but the trend of improving pipeline safety did come to an end. The end of a learning process is well known in the literature on \textit{organizational learning}. The theory of \textit{learning curves} is built on the assumption that the efforts of an organization to optimize a performance measure will end up looking like a distinct curve that initially falls quickly before learning declines \citep{Argote2013-1}. Figure 2 shows that pipeline safety follows the same pattern.

{\noindent}\dotfill

\centerline{Insert Figure 2 about here}

{\noindent}\dotfill

\begin{figure}
	\caption{Pipeline safety improvements at the industry level for refined petroleum pipelines}
	\centerline{\includegraphics{../illustrations/population_learning_5.png}}
\end{figure}

The literature on organizational learning largely focuses on three issues: improvements of performance measures other than labor hours per unit of production (which the early literature focused on), knowledge transfer or vicarious learning, and organizational forgetting \citep{Argote2013-1}. This research does speaks to the first point, learning toward a diverse set of outcome variables, but it is not the focus of this work and the evidence is already quite compelling that organizations learn and improve any performance measures they track. Evidence is also conclusive that organizations learn vicariously from other similar others \citep{Kim2007}. The last point, organizational forgetting, has received less attention to date. Organizational forgetting paints a decisively less straightforward picture of organizational learning than the literature that precedes it. The concept acknowledges that learning is not always linear, and highlights that organizational dynamics have a bearing on knowledge \citep{Argote2013-3}.

If we take the learning curve seriously, we would expect that rapid improvements of performance measures are the exception, rather than the norm. Not on the organizational level, but on the level of a business unit or individual products or technologies, we would expect organizations to only spend a short time in a period of rapid improvements. Most of the time is spent in the tail end of the learning curve, where only marginal improvements are achieved, if any.\footnote{Conversation with colleagues always yields a lot of anecdotal evidence for this.} Organizational learning offers little in the way of explaining the process by which learning comes to a standstill.The literature currently has a a strong focus on \textit{organizational knowledge} \citep{Bingham2011}, from which we can derive two possible explanations. Either organizational knowledge eventually reaches a point of saturation, where the organization has attained an almost perfect command of the process. Or organizational forgetting could be a mechanism of convergence in learning. If an organization was to forget at the same pace as it is learning, that could give the impression of stagnation. This research will discuss the question head on: \textit{how does the convergence of a performance measure take place?}

This research takes a mixed-methods approach to uncover what happens inside organizations after the phase of rapid learning. The qualitative section takes a deductive approach. The qualitative data directly reveals that and how learning takes place in the pipeline industry, which is in the "long tail" of the learning curve. The quantitative section explains how the observation of learning in the pipeline industry is compatible with an absence of aggregate improvements of the performance measure. The qualitative section takes advantage of the well-established link between failure and learning \citep{Kim2007, Baum2007}. Failures are an unambiguous form of performance feedback and a strong catalysts for learning. Failure allows organizations to identify inadequate assumptions that underlie their activities, and develop models that better represent the world \citep{Madsen2010}. Pipeline spills act as looking glasses for this research, highlighting the problems that still exist in the management of pipeline safety and magnifying the actions taken by the industry and individual actors alike. The qualitative section is based on a sample of ten recent, significant pipeline spills. The quantitative section uses a dataset of the Pipeline and Hazardous Materials Safety Administration (PHMSA) on pipeline operators and spills. For the period from 2004-2019, this dataset holds information on 6,146 spills, including 2,246 that PHMSA deems significant. This research utilizes the textual descriptions of the spills in conjunction with spill rates to show that pipeline spills do trigger learning, but that pipelines have developed into complex systems that again and again surface unexpected interactions which lead to spills. To process that text data, I employ Topic Modeling \citep{Hannigan2019}. Difference-in-difference is used to show that learning still takes place in the pipeline industry in response to pipeline spills, despite a stagnation of pipeline safety.

Whilst the main contribution of this research is to shed light on the convergence stage of learning that arguably most organizations are in most of the time, the relevance of this research extends beyond just this phenomenon. The learning literature might give off impression that organizational learning is impactful and commonplace, when the reality might be that organizations run into limits all the time. Organizational learning might be the exception rather than the rule. By highlighting the limitations of conventional organizational learning, this research underlines the importance of learning that goes beyond optimizing performance measures. Something that the \textit{organizational knowledge} literature does not capture well is possibility to attain ones goal in a nonlinear fashion, for instance by questioning or abandoning currently held knowledge. With regard to attaining for example the viability or sustainability of a business unit, this research underlines the importance of exploration \citep{March1991}. For example, if society were to decide that the current pollution from pipelines spills is unacceptable, but that instead spill levels should be significantly lower, the trajectory that this research has identified suggests that pipeline operators or energy systems would need to make a radical departure from the current approach. The same could be the case for other systems with significant environmental impacts. Hence, this research also speaks to the discourse on \textit{grand challenges}: rather than highlighting individual contributions, this research alerts us to general trends. Here, literature on nonlinear learning can offern valuable insights \citep[e.g.,][]{Argyris1978, March2010}, as will be laid out in the discussion section.

%By understanding what happens in organizations . But potentially also what we can do about it. Is the process inevitable?

%will contribute of an understanding what we can do about it

%Organizational learning comes down to choices. Firms can either invest in improving existing technology, or develop new technology \citep{March1991}. Investing in the "wrong" technology can lead to technological lock-ins \citep{Levinthal1993}. The actors in the pipeline industry have selected a number of technological solutions to resolve their most pressing issue. When a pipeline spill occurs, the oil quickly infiltrates the soil and seeps into the groundwater.\footnote{The infiltration depth in sand is assumed to be over 10m in the first day alone \citep{Bonvicini2015}.} The environmental degradation caused by oil affects the local environment, and the local populace, too: a 2019 sibling comparison study on oil spills in Nigeria found that in localities that are affected by oil spills, for every 1,000 live births, an additional 38.3 neonatal deaths occur\citep{Bruederle2019}. % Potentially add impact of spills on industry? Stigmatized industry.

%In their fight against pipeline spills, pipeline operators employ a variety of technologies, such as smart pigs, leak detection systems, and SCADA systems. Smart pigs, while traveling through the pipes, utilize electromagnetic flux or ultrasonic probing to assess corrosion or mechanical damages to the pipe \citep{Singh2017-7}. Internal leak detection systems measure the flow of oil at two points A and B to detect any loss in between those points. External leak detection systems detect signs of escaping hydrocarbons, and include acoustic, hydrocarbon, and temperature sensors. \citep{Shaw2012}. SCADA systems are computer systems that allow an operator remotely monitor and operate lines. The operator typically sees on his screen charts of the flow at different points, can open and close valves, and startup or shutdown delivery of oil. Alarms from leak detection systems of the line are also displayed to the SCADA operator.\footnote{Larger pipeline companies operate control centers where all lines in a region are managed. Operators usually operate multiple SCADA systems at once, and more experienced employees supervise the operators. Control centers are operated in formal hierarchy, where for certain operations (such as clearing an alarm), a SCADA operator will require the go-ahead from a supervisor. See \citet{NTSB2012} for an in-depth description of an Enbridge control center in Edmonton as of 2012.}

%The high technology character of leak detection stands in contrast to the experienced reality of pipeline spills. A 2012 study commissioned by the Pipeline and Hazardous Materials Safety Administration (PHMSA) of onshore pipeline spills that occurred over a 19 month period, SCADA systems assisted in less than 25\% of cases with the detection and confirmation of the spill \citep[p. 3-33]{Shaw2012}. In only 17\% of cases was the operator or SCADA system listed as the initial identifier of the leak, while the public or emergency responders identified 30\% of leaks \citep[p. 3-39]{Shaw2012}. Why do the great learning efforts by pipeline operators fail to deliver the safety improvements that one would expect to see? A 2012 report prepared by the National Transportation Safety Board (NTSB) on the Kalamazoo River oil spill provides a good starting point for understanding the problem. A regional manager of Enbridge is quotes as saying: "...I'm not convinced [that there is a problem]. We haven't had any phone calls. I mean it's perfect weather out here--if it's a rupture someone's going to notice that, you know and smell it" \citep[p. 100]{NTSB2012}.

%This chapter uses the quantitative data from PHMSA to demonstrate how existing problems are addressed, following major spills that catch the attention of the industry, the regulator, and the public. That empirical observation is contrasted with the character of the two challenges that remain: (1) as holes are plugged, new unique sources of spills, for example climate change-related weather changes, emerge. (2) Both the "human factor" and the "organizational factor" are pervasive factors that yet to be completely eliminate as sources of error in any context. Overall, the geographic, technological, and organizational complexity of pipelines have led to the current situation, a quasi-standstill in the sector for refined oil. Crude oil pipelines on the other hand still have some potential for improvements, as simple and fundamental problem of this sector-- the corrosiveness of the commodity-- is addressed through new coatings, and cathodic protection.

%The quantitative section of this chapter uses a sample consisting of the 100 largest operators in the pipeline industry over the period from 2004 through 2019. The data that is available from PHMSA is matched and supplemented with data from Compustat. This quantitative section focuses on improvements over time in organizations affected by a specific source of incidents, and a reduction in certain causes of spills. Some qualitative data supplements the quantitative analysis by showcasing the processes of population level learning \citep{Miner1999}, especially for crude pipelines. The qualitative section then uses archival data to explore a sample of 15 major pipeline spills since 1986 to contrast the specificity of learning in the quantitative analysis with the complexity of the systems that the incidents occur in, and the complex interactions that lead to spills. The sample of 15 spills includes the top three spills with regard to spill volume, net loss (spill volume minus volume recovered), number of injuries, number of fatalities, and property damage. This sampling method ensure both a variety in the type of spills, and a good availability of archival data.

%This chapter contributes to the literature on knowledge-based learning by exploring the topic of a bottomed-out learning curve through raising the issue of aggregate and specific learning. The chapter also contributes to the debate on industry resource use: it discusses both the historical development and the potential future reduction (or lack thereof) of an industry's environmental footprint. Whereas in the past, improvements were made through incremental learning in the form of development of new technology, the analysis suggest that further improvements may only be possible through bold, maybe costly new choices, including a change of industry for some companies.

%Our qualitative analysis reveals that pipeline spills have all the hallmarks of normal accidents in complex systems \citep{Perrow1984}. Almost no two serious spills are alike, and the causes are as complex as the diverse political and physical environments that is the United States. Here, organizational learning is at an impasse. When efforts is made, and learning takes place, why do we not observe corresponding results? \citet{Levitt1988} propose that there are limitations to learning by doing. In those cases, learning cannot be disaggregated into its components. Instead, we need to look at the technological choices and determine whether an organization or industry has ended up in a competency trap \citep{Levitt1988}. An important factor for diagnosing this issue are feedback mechanisms: at the population level, is the problem diagnosed, or not? If in an industry the lack of learning goes unnoticed or is not addressed on a population level, even if learning takes place on a case-by-case basis, aggregate learning may not take place {MarchOlsen}.   

%With this article, we provide an additional perspective to the learning literature. Our interpretation of the data puts into question the notion of aggregate improvements through incremental, smaller scale learning. Instead, there are more substantive, technological improvements to be made, that sometimes cannot be attained through regular learning mechanisms. In those cases, a big picture perspective on the problem or goal is necessary to made a difference.

%   Other contribution: write something on learning that fills the gap created by publication bias.

%	Hints at a problem in the learning literature. Focusing on incremental improvement.

%   Claim quantitatie data as part of initial qualitative research? We looked at 10 major spills and x years of spill data, to understand how population-level learning works. Also, claim data on population level learning as part of the research.

%	Qualitative research approach--select spills--move through spills until motives saturated.

%	Limited ability to triangulate--but that is fine because our qualitative results are quite robust, and not too complex--little chance they are wrong.

%	Another omission that we decided on for this article are flaws in the environmental feedback mechanism, akin to those predicted in \citet{March1975}. Pipeline operators are well-insulated from the consequences of their actions, as the regulator (PHMSA) is understaffed and generally gives pipeline operators the benefit of doubt [provide evidene from NTSB].


	
	\subsection{Chapter 2: Theoretical Foundations of Learning}

This literature review focuses on organizational learning. The purpose of this chapter is to shed light on the intricacies of learning in the pipeline industry from a theoretical perspective. The first section summarizes the literature of the knowledge-based approach. In particular, the section emphasizes the strength of the knowledge-based approach, which accurately describes the accumulation of a knowledge stock that does take place in an organization when the goal is more or less clear and the environment stable. Further, This first section summarizes some of the other accomplishments of the literature that roughly fall into the knowledge-based stream, specifically its predecessor the learning curve literature, as well as vicarious learning, and population level learning.

The second section summarizes the behavioral approach to organizational learning. In particular, this section highlights the gaps left by the knowledge-based approach that may be filled by work in the behavioral space. The behavioral approach appreciates the nonlinearity and messiness of learning. Significant learning can results from individuals or groups organization that seek for the organization to break with convention \citep{Argyris1978}. This form of learning poses a challenge to the knowledge-based literature, because the changes that take place fall outside the normal rubric of improvement (as exemplified by learning curves). Other examples of these qualitatively different dimensions of learning in the literature are exploration and exploitation \citep{March1991}, and high intellect vs. low intellect learning \citep{March2010}.

Building on the review of the two approaches to learning, this literature review then discusses two critical issues, which become particularly important in the context of pipeline spills. First, in light of the leveling off of the learning curve \citep[which in the pipeline industry takes the shape of a "baseline" of spills, or "normal accidents"][]{Perrow1984}, this section addresses the merits of less complex technologies. The learning literature is implicitly technocentric, but when efficiency does not take the primacy--as is the case when dealing with toxic chemical such as oil-- rather than adding and improving technologies, scaling back and relying on less complex technologies may be a better way to go.

Second, in light of the damages that have been brought about by the petroleum industry, it is  timely to take a critical look at the modernistic assumptions of the learning literature. The learning literature should confront the fact that new technolgies also routinely bring about new problems \citep{Beck1992}. To stay with the theme of this dissertation: pipeline spills (and climate change) are problems that did not exist before the petroleum industry came into being. Acknowledging the conjunction of technology and risks is not meant to take away from the recognition of its merits. Yet, we encourage a debate about the tendency of the literature to equate "newer" with "better" rather than "different".

% multiple learning curves either fall into a learning progress toward a more abstract goal, or in a less modernistic conceptualization, learning [verb] when the organization or population turns its head toward a new goal.
	
	\section{Chapter 3: The Green Black Gold Blues. Diffusion of greenwashing in the pipeline industry}

\begin{quote}
	'[W]e are building a pipeline that is state of the art and will be the safest pipeline  ever build.' -- TransCanada President \& CEO Russ Gerling on the Keystone Pipeline\footnote{\url{https://youtu.be/ctx0H8XR51s?t=127}, accessed 2020-08-23.}
\end{quote}

\begin{quote}
	Although pipeline technology has improved, new pipelines are subject to proportionally higher stress as companies use this improved technology to maximize pumping rates through increases in operational pressures and temperatures, rather than to use this improved technology to enhance safety margins. -- Excerpt from a technical report that challenges the environmental impact assessment for the Keystone Pipeline \citep[p. 4]{Stansbury2011}
\end{quote}

For decades, activists have called out corporations for not walking the talk. In 1992, Greenpeace warned that the most powerful corporations' rhetorics mostly serve to distract the public, while these corporations fight off liability and accountability in the judiciary and legislative arena \citep{Bruno1992}. To describe this phenomenon of corporations building a false green image, grassroot movements have coined the term "greenwash". Greenwashing describes "any communication that misleads people into adopting overly positive beliefs about an organization's environmental performance" \citep[p. 225]{Lyon2015}. Generally, corporate communication with stakeholders is driven by their goals and interests. For instance, what corporations stay quiet about is as important as or more important than what they do disclose \citep{Kim2015}. Hence, what corporations say cannot be taken at face value. The greenwashing literature researches this problem with regard to environmental performance. The most common style that has been explored in the literature is disclosure of only positive information on environmental performance and withholding of bad news \citep{Lyon2011}. To date, research has focused on different types of greenwashing, their prevalence, and performance implications for corporations \citep{Marquis2016, Ramus2005, Seele2017, Kassinis2018}.

The existing research shows that greenwashing is a phenomenon that plays out not only at the level of individual organizations. In empirical research, to control for industry effects has become the norm \citep[e.g.,][]{Ramus2005, Marquis2016, Du2015, Testa2018}. The necessity to control for the industry indicates that there are important processes taking place within industries. Standards and research insights that are shared across an industry can act as templates for greenwashing, and organizations copy each other's greenwashing strategies. The greenwashing literature has yet to cover these processes. Under the watchful eye of stakeholders, entire industries such as mining, agriculture, or the energy sector have come under suspicion across the board and need to constantly put in efforts to legitimize their business models. In cases such as these, industry could even surpass organizational factors as a predictor for greenwashing. Hence, a discussion is overdue on the question: \textit{How does the industry affect organization's propensity to greenwash?}. Assisting with this question can other literature on inter-industry processes \citep[such as][]{Maguire2009, Hardy2020}.

To empirically demonstrate how industry-specific greenwashing strategies are diffused, this research turns to an industry where greenwashing has taken a very peculiar form. The pipeline industry uses the veil of engineering to present itself as safe, and pipeline technology as perfectly controllable, despite pipeline spills being a regular occurrence in the US. The public repository of the Pipeline and Hazardous Materials Safety Administration (PHMSA) holds data on both individual operators' pipeline miles and the volume of crude and refined petroleum transported. Further, the repository offers a description of and quantitative data on each minor and significant pipeline spills that has occurred in the US. The analysis of text data for this research relies on Natural Language Processing (NLP)--specifically, Topic Modeling--to determine spill causes and technology trends \citep{Hannigan2019}. The descriptions of individual spills reveal the shortcomings that individual operators exhibit in terms of pipeline safety. This data is matched with text data on pipeline safety strategy obtained from annual reports or, where available, safety reports. Annual or safety reports provide insight into the strategic plans and actions of operators. Next, data on headquarter location and executives' connections (BoardEx) surfaces networks within the industry. Finally, documents by industry-level actors such as the American Petroleum Institute (API) or the PHMSA unearth the latest industry trends. Greenwashing is given where non-substantive industry trends, rather than the operator's safety problems, determine individual operators strategic plans and action. By using operators' spill frequency and volume over time, we can ensure that effective measures are not accidentally flagged as non-substantive.

By researching greenwashing in the pipeline industry in the form of non-sustantive strategic plans and actions in the pipeline industry, this research expands the greenwashing literature. The empirical data reveals the flow of information within the industry, and the contribution of intra-industry networks to greenwashing. Greenwashing in the form of engineering and technology-centric communication also represents an addition to the literature. This form of greenwashing is particularly insidious, because an observer needs to first penetrate a layer of engineering and technology lingo, before the underlying issue can be surfaced. The addition of this form of greenwashing to the literature could help direct attention to other industries that have developed sophisticated forms of greenwashing which may be impossible for laymen to discern. Polluting industries that make intense use of new technologies are likely candidates to exhibit this form of greenwashing, for example chemistry, engineering, and construction. Exposing the role of industry-level actors such as the API, and an industry-wide propensity to greenwash also has relevance outside academic circles. Where the industry plays a role in greenwashing, policy makers and activists that seek to reign in greenwashing need to take a more systemic view, and target industry-level actors, or industries as a whole. On a related note, this unique cross-level research, which spans from the industry down to individual spills, also contributes to the literature on social-ecological systems \citep{Reyers2018}.\footnote{More recently, the need for cross-level research has also been voiced repeatedly during the ARCS Online Seminar Series and Ivey Sustainability Salon. For instance, at the Ivey Sustainability Saloon session on July 16, 2020, Tima Bansal to Nicholas Poggioli: "If the firm is at one level, one could argue that the eco-system is a different level in which many actors interact. And, arguably, Sustainable Development is a macro-level concept (system of actors)."}

%Many corporations engage in greenwashing. So what, one may say, there is always a competitor one can buy from. But not every industry has an incumbent that can be trusted to act responsible and whose products are widely available. Especially in concentrated industries, finding a "good egg" may be difficult. Greenwashing in these industries can become an issue that is pervasive, that is "part of the culture", and that is hard to eradicate. This research takes a look at one such industry where greenwashing is pervasive to understand how greenwashing spreads across this industry. From the pipeline industry, the this research generalizes to industry-wide diffusion and proliferation of greenwashing.
%
%The existing literature discusses multiple different styles of greenwashing. Their commonality is that they encompass "any communication that misleads people into adopting overly positive beliefs about an organization's environmental performance" \citep[p. 225]{Lyon2015}. Selective disclosure describes a process whereby a corporation discloses favorable information about its environmental performance, whilst withholding negative information, to attain a positive image. When implemented correctly and in the 
%absence of a harsh, reliable punishments for greenwashing the market may reward this behavior with a higher valuation \citep{Lyon2011}. Usually, it is assumed that organizations greenwash to meet stakeholder expectations, to give the impression of transparency, or simply out of opportunism \citep{Kim2015, Marquis2016}. The definition of greenwashing does not prescribe a specific mechanism that would have to be at play for an activity to qualify as greenwashing.
%
%Pipeline operators use often new and unproven technology to window dress their performance and gain permits for the construction of new pipelines. The operators advertise their use of technologies that are "safe"--that can be proven to fulfill their function in a lab setting or simulation--but that have not served to drive down the number or volume of oil spills since the turn of the millenium (see section "Introduction"). The reason for this shortcoming is that these technologies do not address the causes of spills. We will denominate these technologies in this research as not issue-oriented. The industry maintains a strong claim to safety. Pipelines, unlike oil-by-rail, are asserted to be safe, by merit of the physical principles that they operate on. Further, modern technology, as employed to improve pipeline safety, is asserted to be perfectly controllable. Typically, greenwashing in the pipeline industry falls into the realm of nonmarket strategy. This becomes most obvious when documents are specifically written to obtain a permit or attain another goal. But greenwashing is also present on corporate websites, annual reports, and publications by industry associations like the American Petroleum Institute. These documents emphasize safety initiatives, and gloss over recent spills, or emphasize spill response and remediation efforts. This myth of safe and controllable pipeline technology has also allowed the pipeline industry to attain broad public and political support in many states.\footnote{In Louisiana for example, anybody who protests on or near pipelines faces up to five years in prison, hard labor, and a \$1,000 fine. \url{https://www.propublica.org/article/how-louisiana-lawmakers-stop-residents-efforts-to-fight-big-oil-and-gas}, accessed 2020-08-19}
%
%This research takes a longitudinal approach to greenwashing and draws on data for the period form 2002 to 2019 from multiple datasets to examine greenwashing in the pipeline industry. (1) The Pipeline and Hazardous Materials Safety Administration's (PHMSA) dataset on pipeline spills and (2) on pipeline miles operated by individual operators and their utilization allows for the identification of pipeline spills and their causes. (3) Documents by the American Petroleum Institute (API) and other industry-level actors allow us to monitor the emergence of new, non-substantive technologies that fall into the realm of pipeline safety, but do not address common causes of pipeline spills. (4) Public documents by pipeline operators, such as annual reports or dedicated safety reports show the diffusion of these technologies and their use for greenwashing. (5) Data on executives in the pipeline industry, obtained from BoardEx, reveal the web of connections along which rhetorics are diffused in the industry. Finally, (6) fines levied on pipeline operators by the PHMSA, (7) post-spill statements by pipeline operators, that reveal their rational and perceived errors (albeit only ex ante), and (8) accident reports by the National Transportation Safety Board (NTSB) allow us to qualitatively assess the divergence between technological developments and spill causes.
%
%To demonstrate the degree of greenwashing in the pipeline industry, this research determines the degree to which pipeline operators' communication with the external environment is driven by industry-level trends rather than by the individual operators' recent spills. This empirical examination also gives us an understanding of how greenwashing can become commonplace in an industry. The discussion of technology that is not issue-oriented also draws attention to other forms of greenwashing that front technology but miss the issue at hand. Other example include fracking, carbon offset, and possibly carbon capture and storage. This research may also allow stakeholders and activists to better target more insidious forms of greenwashing.
%
%\subsection{Greenwashing in the pipeline industry}
%
%Greenwashing is quite common in the American pipeline industry, and the fossil fuel sector in general \citep{Kassinis2018}. The industry extensively uses the language of engineering in its rhetoric. Its communication with the outside environment is typically carried out by engineers (e.g., in feasibility studies or environmental impact analyses). American Petroleum Institute (API) engineers define standards and explores new technologies for pipeline operators. The API simultaneously lobbies against climate action. It is difficult for the "receiving side" of the communication to see through the greenwashing strategy of the pipeline industry, unless they also have specialized engineers at their disposal.\footnote{For a successful attempt, see e.g., \citet{Stansbury2011}.} The challenge is not that the claims made by engineers in the pipeline industry are factually incorrect, but that they do not address the issues that most frequently lead to pipeline spills (e.g., they involve not issue-oriented technologies). The language of engineering that documents are dressed in prevent most readers to get through to that issue
%
%By taking advantage of the language of engineering, pipeline operators gain access to a wide array of greenwashing strategies. The obvious is a stated commitment to pipeline safety, dressed in the language of engineering, but without substantiating the action to be taken. In other cases, greenwashing takes a more insidious approach, wherein the language of engineering and pipeline safety will conceal the primacy of business interests. For instance, the language and engineering can be used to misrepresent investments with a financial interests as primarily motivated by safety, or to obtain objectively verifiable data that cannot be refuted by stakeholders that are laymen. The following are two examples: (1) after a 2010 oil spill that polluted the Kalamazoo River in Michigan, Enbridge agreed to spend at least \$110 Million to improve pipeline safety.\footnote{\url{https://archive.epa.gov/epa/newsreleases/united-states-enbridge-reach-177-million-settlement-after-2010-oil-spills-michigan-and.html}, accessed 2020-08-19.} Enbridge successfully translated this settlement into an investment by using the money to increase its pipeline capacity and transport more oil along the same route. The residents that had been affected by the spill lost out against Enbridge a second time: once again they had their lifes disrupted, this type by construction work in their backyard, sometimes on a very short notice. Enbridge successfully circumvented the need for an environmental impact assessment by replacing the pipeline in short sections.\footnote{For brief summary of events, see \url{https://www.youtube.com/watch?v=IAR7z76KWj8}, accessed 2020-08-08.}
%
%(2) In 2006, in a confidential document that was later leaked, a consulting company on behalf of the TransCanada Corporation made two claims. If the Keystone Pipeline was to be constructed the operator could using the latest technology detect a large spill in as little as 9 minutes, and any spill over 50 barrels would only occur once every seven years \citep{Consulting2006}. This claim was not testable at the time but did convince the regulator to greenlight the construction of the pipeline. The pipeline began operation in 2010, and as of 2020 had experienced 5 spills of over 50 barrels,\footnote{See \url{http://boldnebraska.org/keystone-pipeline-spill-history/}, or \url{https://julianbarg.shinyapps.io/incident_dashboard/}, accessed 2020-08-08.} including one case where the spill continued for 20 minutes after the affected landowner who had discovered the spill had called in\footnote{\url{https://www.thedickinsonpress.com/business/energy-and-mining/4004561-5-years-after-spill-rancher-and-pipeline-junkie-still-has}, accessed 2020-08-08}. The problems of the Keystone Pipeline are symptomatic of spill detection technology: a sensitive system can detect small spills but will also produce many false positives. Thus, the real challenge is actually the far more complex one of managing the safety culture, since personnel can easily become desensitized by frequent false alarms and even safety drills.\footnote{See e.g., \url{https://www.ntsb.gov/investigations/AccidentReports/Reports/PAR1201.pdf}, p. 101.} For a discussion of the role of the consulting firm in misleading the regulator in that case, see \citet{Stansbury2011}.
%
%\section{Assessing greenwashing in the pipeline industry}
%
%In order to empirically assess greenwashing in the American pipeline industry, this research focuses on two kinds of greenwashing. The first is non-substantive promises made. These are commitments to pipeline safety without details on action, or without the organization following through. This version of greenwashing can be coded by hand, as long as comparable documents are available across time. The second aspect are technologies that are not issue-oriented. Referrals to these technologies, we can capture by using Natural Language Processing (NLP): either through keyword searches, or by utilizing topic modeling.
%
%Since the quality of the text data will be the key to obtaining meaningful results, constructing a good sample is of the essence. A list of the largest pipeline operators can be extrapolated from the Pipeline and Hazardous Materials Safety Administration (PHMSA) dataset. To obtain good quality text data, my sampling strategy focuses on the largest players in the industry, and completeness across time. Where annual reports or similar documents cannot be obtained from central sources such as Mergent Archives, the SEC, or \url{www.annualreports.com}, the documents are collected as much as possible by hand from corporate websites. For the empirical section of this research, the two different types of greenwashing (non-substantive and technology-based) are tracked across organizations and across time. We obtain evidence that teh communication does in fact constitute greenwashing by comparing the diffusion of technology in rhetorics with actual spill rates and volumes. The empirical test then shows that the communication with the external environment on pipeline safety is determined by networks of diffusion for greenwashing much more than by recent failure modes, by substantive safety issues that could be addressed. To show that mismatch between communication and issues, this research takes advantage of both non-substantive communication and of the different types of technologies that are applied but that are not issue-oriented. These types of greenwashing we can track across the company documents and industry-level documents, and compare that data to the network data available from BoardEx. The importance of industry-level actors comes even more into focus when we turn to instances of deregulation in Louisiana and Texas. These instances of deregulation should act as exogenous shocks for greenwashing. We can expect that following these exogenous shocks the share space given to technologies that are not issue-oriented or non-substantive communication should become even more widespread.
	
%	\section{Conclusion}

Over the last five years, some debate has taken place in the management literature on our use of decreasing natural capital, under the umbrella of grand challenges. This dissertation contributes to that literature, by raising the issue of chemical pollution, and alleviation of impacts. In lieu of a great lot of literature on this issue in the management literature, this dissertation turns to \textit{organizational learning}, a literature that has discussed related problems for some decades.

The purpose of this work is twofold. On the one hand, it brings attention to the great knowledge stock that exists in management research research on organizational learning and the behavior of organizations, which may benefit further discussions on alleviation of impacts, and use of decreasing natural capital in general. On the other hand, by bringing attention to the systemic issues that are highlighted by the grand challenges literature, this dissertation also provides a new direction for research on organizational learning. An empirical analysis of learning on the organization level would have likely overlooked the lack of aggregate learning that exists in parts of the pipeline industry (with regard to refined petroleum pipelines). The grand challenges literature here provides an important impulse to revisit some assumptions.

% 	\input{sections/timeline}

\bibliography{bibliography}

\end{document}