\documentclass[12pt, man, natbib]{apa6}
\usepackage[USenglish]{babel}
\usepackage{setspace}
\usepackage{hyperref}

\title{Green is the New Black? Temporal Dimensionality of Greenwashing}
\shorttitle{Genuine Intentions}
\author{Julian Barg\\barg.julian@gmail.com}
\affiliation{Ivey Business School}
\setcitestyle{authoryear, open={()},close={)},citesep={,},aysep=}


\abstract{}


\begin{document}
	
	\maketitle
	
%	\singlespacing
	
	\section{}
	
	How does greenwashing play out over time? The greenwashing literature has taken a cross-sectional approach to a problem that is characterized by its time dimension \citep{Lyon2015}. An organization may move away from greenwashing, because of associated costs, or in a recoupling process. Or when there are positive-- or at least no negative--consequences to greenwashing, this may reinforce the greenwashing behavior. One could also imagine a strategy of interspersing greenwashing with effective measures to fend off critics, or offsetting greenwashing with investments in new, unproven technology. In the following, I suggest some hypotheses related to the development of greenwashing over time that are testable in the pipeline industry. I also give a brief introduction into greenwashing in the context of the pipeline industry.
	
	This research will takes a longitudinal approach to greenwashing and draws on data for the period form 2002 to 2019 from multiple datasets to examine greenwashing in the pipeline industry, including: (1) The Pipeline and Hazardous Materials Safety Administration's (PHMSA) dataset on pipelines operated by American pipeline companies and its accompanying dataset on pipeline spills, (2) statements by the pipeline operators on their plans with regard to pipeline safety, taken from their annual reports, (3) news and social media coverage of pipeline operators which calls out greenwashing, (4) fines levied on pipeline operators by PHMSA, and (5) post spill statements by pipeline operators, that reveal their rational and perceived errors (albeit only ex ante).
	
	The literature suggests that greenwashing can become commonplace under some circumstances. For example, businesses can find that it is cheaper to send the "right" signal and obscure actions rather than to address concerns as expectations with regard to sustainability increase within their industry \citep{Delmas2011}. In the pipeline industry, greenwashing is rampant. The industry advertises pipelines as a safe way of transporting oil and gas even as all pipeline operators regularly experience pipeline spills, with sometimes catastrophic consequences. % Give intro to industry
	Common strategies are non-substantive commitments to pipeline safety, adoption of unproven technology to fend off critics and the regulator, and investments that are disguised as safety upgrades. By gaining an overview over the greenwashing behavior in this specific industry, we also gain a starting point for discussing greenwashing on a wider scale (beyond individual companies), and the impact we might expect on a larger scale.
	
	This research will open the discussion on the longitudinal development of greenwashing. Do firms recouple? Does deregulation lead to more greenwashing? Does public attention foster a stronger coupling of rhetoric and action? The context of pipeline safety provides a suitable context to put these research questions to the test. If firms generally do not recouple, we would expect greenwashing behavior to sustain over time. To test the effect of deregulation, we can take advantage of regulatory changes in Louisiana and Texas as exogenous shocks. Does greenwashing generally increase under a less strict regulatory regime? Or does this environment make businesses feel less obliged to greenwash? And public attention to the industry came about in the aftermath of the Deepwater Horizon oil spill, which drew attention to crude oil pipelines more than to refined petroleum and gas pipelines. In addition, understanding the temporal dimensionality of greenwashing could also allow us to preempt greenwashing better. In particular, finding out the role of the regulator, and the frequency of recouping would be strong indicators of where we should focus our efforts.
	% Explain regulatory capture
	
%	Across time so we can preempt these tactics
	
	\section{Greenwashing in the pipeline industry}
	
	Greenwashing is quite commonplace in the American pipeline industry. The industry extensively uses the language of engineering in its rhetoric. Its communication with the outside environment is typically carried out by engineers (e.g., in feasibility studies or environmental impact analyses). American Petroleum Institute (API) engineers define standards and explores new technologies for pipeline operators. The API simultaneously lobbies against climate action. This composition of the industry, with workers (e.g., workers who look to pivot away from rig work or undergrad engineering graduates) and managers at most levels of the organizations, but engineers in charge of communication allows for a decoupling of different parts of the organization, and communication on the basis of "clean" decontextualized engineering such as new, unproven technology.
	
	Pipeline operators have a further set of greenwashing strategies at their disposal. The obvious is a stated commitment to pipeline safety without substantiating the action to be taken. In other cases, greenwashing takes a more insidious approach, wherein the language of engineering and pipeline safety will conceal the primacy of business interests. For instance, the language and engineering can be used to misrepresent investments with a financial interests as primarly motivated by safety, or to obtain "hard data" that cannot be refuted by stakeholders that are laymen. The following are two examples: (1) after a 2010 oil spill that polluted the Kalamazoo River in Michigan, Enbridge used the money it was obligated to spend on safety to instead upgrad its pipeline capacity in the area. The process of installing a new line with a larger diameter meant that many of the residents who were affected by the spill once again had their lifes disrupted by construction work in their backyard, sometimes on a very short notice.\footnote{For brief summary of events, see \url{https://www.youtube.com/watch?v=IAR7z76KWj8}, accessed 2020-08-08.}
	
	(2) In 2006, in a confidential document that was later leaked, a consulting company on behalf of the TransCanada Corporation made two claims. If the Keystone Pipeline was to be constructed, the operator could, using the latest technology, detect a large spill in as little as 9 minutes, and any spill over 50 barrels would only occur once every seven years \citep{Consulting2006}. This claim was not testable at the time but did convince the regulator to greenlight the construction of the pipeline. The pipeline began operation in 2010, and as of 2020 had experienced 5 spills of over 50 barrels,\footnote{See \url{http://boldnebraska.org/keystone-pipeline-spill-history/}, or \url{https://julianbarg.shinyapps.io/incident_dashboard/}, accessed 2020-08-08.} including one case where the spill continued for 20 minutes after the affected landowner who had discovered the spill had called in\footnote{\url{https://www.thedickinsonpress.com/business/energy-and-mining/4004561-5-years-after-spill-rancher-and-pipeline-junkie-still-has}, accessed 2020-08-08}. The problems of the Keystone Pipeline are symptomatic of spill detection technology: a sensitive system can detect small spills but will also produce many false positives. Thus, the real challenge is actually the far more complex one of managing the safety culture, since personnel can easily become desensitized by frequent false alarms and even safety drills.\footnote{See e.g., \url{https://www.ntsb.gov/investigations/AccidentReports/Reports/PAR1201.pdf}, p. 101.} For a discussion of the role of the consulting firm in misleading the regulator in that case, see \citet{Stansbury2011}.

	\section{Assessing greenwashing in the pipeline industry}
	
	In order to empirically assessing greenwashing in the American pipeline industry, we will focus on two aspects. First, non-substantive promises made. These are commitments to pipeline safety without details on action, or without the organization following through. This version of greenwashing can be coded by hand, as long as comparable documents are available across time. The second aspect are referrals to technology. These referrals we can capture by using Natural Language Processing (NLP): either through keyword searches, or by utilizing topic modeling.
	
	Since the quality of the text data will be the key to obtaining meaningful results, constructing a good sample of the essence. A list of the largest pipeline operators can be extrapolated from the Pipeline and Hazardous Materials Safety Administration (PHMSA) dataset. An appropriate sampling strategy focuses on the largest players in the industry, and completeness across time. Where annual reports or similar documents cannot be obtained from central sources such as Mergent Archives, the SEC, or \url{www.annualreports.com}, the documents are collected as much as possible by hand from corporate websites.
	
	Greenwashing would be given in a case when non-substantive commitments to pipeline safety, or referrals to unproven technology coincides with a lack of safety improvements.\footnote{We recognize that this is an imperfect measure of greenwashing, since organizations could take action on pipeline safety without announcing them in detail. However, to identify decoupling to this level of detail is beyond the scope of this study.} Reading and categorizing the reports using NLP gives us two different ways to capture greenwashing. The stability of this relationship across time then gives us an indication of the persistence of greenwashing behavior across time. As mentioned above, exogenous shocks further allow us to measure the impact of the regulator and the public on greenwashing behavior across time. Instances of deregulation in Louisiana and Texas allow for testing the importance of governance, whereas the Deepwater Horizon oil spill directed attention toward crude oil production in the Gulf of Mexico, and associated pipelines. Finally, post spill statements by pipeline operators provide us with a qualitative robustness checks. These statements by organizations are used to "come clear" about the errs that lead to a spill. In particular, statements in the form of "we tried our best, but..." reveal aspects of pipeline safety that the organization knows to not work but did not have a motivation to reveal beforehand.
	
%	How many greenwash? How many obscure?
	
%	Unless called out
	
%	In this article, we will take a longitudinal approach to analysing greenwashing, to uncover several points where time comes into play. (1) Time trends of greenwashing. (2) Recoupling where organizations make good choices. (3) Claims going bust, complicates the detection of greenwashing.
%	
%	Politics--once you have the approval, you're in.
%	
%	Incidents--can reveal a genuine introspection.	

\bibliography{bibliography}

\end{document}