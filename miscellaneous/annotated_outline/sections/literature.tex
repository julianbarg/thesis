	\section{Lit review}
	\begin{paracol}{2}
	\begin{itemize}
		\item What is required of the appraisal according to \citet{George2016}
			\subitem Raises 8 points on different levels, actors, multilevel etc.
				\subsubitem (abbreviated)
		\switchcolumn My context offers insights into most of what \citet{George2016} calls for scholars to research (see note). Their editorial is a great starting point to highlight some of the features of my dataset, my intentions going into the research, and the research agenda that I pursue (this last point in the conclusion).
		\switchcolumn*
		
		\item To skip to solution is incomplete!
			\subitem E.g., \citet{Ferraro2015, Slawinski2015}
			\subitem Focus on identify sustainable companies $\rightarrow$ miss industry-wide trends
		\switchcolumn \citet{Ferraro2015} in particular seems based in wishful thinking more than in careful observation and analysis. I want to use this backdrop to tease out the benefits of my analysis. I forgo bold claims of being able to identify sustainable organizations, but I obtain a more convincing analysis.
		\switchcolumn*
			
		\item In line with broader literature \citep{Reyers2018}
		\switchcolumn I see behind \citet{George2016} a larger trend of analysis of social-ecological systems outside of management research, especially in other social sciences, but there is interest from interdisciplinary outlets and natural sciences also \citep[e.g.,][]{Nature2018}
	\end{itemize}
	\end{paracol}
	
	\subsection{Organizations acquiring knowledge}
	\begin{paracol}{2}
	\begin{itemize}
		\item -- 
		\switchcolumn The rest of this section is design to give an introduction to organizational learning (and introduce some of my ideas on that literature).
		\switchcolumn*
		
		\item Learning captures some required elements such as levels and interaction
		\switchcolumn I will explain here why learning is actually a good fit with the laundry list of elements to analyze that \citet{George2016} provides. Aka "we don't need to reinvent the wheel". Systematically cover all (applicable) points.
		\switchcolumn*
		
		\item First stream looks at one outcome variable over time
			\subitem Learning curves/learning from experience
			\subitem Multiple outcome variables would be better, but good start?
		\switchcolumn This is where I clarify why we turned to the learning literature in the first place. There is one variable, and an interest on part of some actors to improve it (although technically it's just a secondary goal). Again, this is about the fit between context and theory.
		\switchcolumn*
			
		\item Different mechanisms identified on different levels
			\subitem Learning from failure
			\subitem Vicarious learning (institutional context)
			\subitem Industry-level learning (coordinating architecture, multilevel actions)
		\switchcolumn Only learning from failure is where I really have interesting insights to offer, but to extend this list to all applicable forms of learning means being able to tell a more coherent story to the sustainability folks, and having a better justification for applying a learning lens.
		\switchcolumn*			

		\item Build in progression assumption without justification
		\switchcolumn The \textit{knowledge approach} has some feature has this questionable assumptions built in. Following the pennant of "different not better" I transition to the routines approach.
		\switchcolumn*
	\end{itemize}
	\end{paracol}

	\subsection{Organizations developing routines}
	\begin{paracol}{2}
	\begin{itemize}
		\item --
		\switchcolumn This section fits nicely in terms of ideas, and in terms of what would be necessary to make headway with regard to pipeline spills, but it does not related to the empirics I propose.
		\switchcolumn*
				
		\item Second order \citep{Argyris1978} and high-intellect learning \citep{March2010} replace the learning curve.
			\subitem Rather than marginally improving a process
			\subitem The process is questioned, better process created
		\switchcolumn A discussion of the data on the pipeline industry in the context of this form of learning may lead the reader to discover herself or himself that a fundamental questioning of processes from within the pipeline industry has not been carried out. Highlights the path-dependence that comes with a process of marginal improvements.
		\switchcolumn*
		
		\item Routines view further capturing more elements
			\subitem Aspirations
		\switchcolumn Aspiration level is one of the few ideas from the routines view I could possibly capture empirically.
		\switchcolumn*
				
			\subitem Politics as barrier
				\subsubitem BTOF \& reliability
		\switchcolumn In the discussion, the concept of reliability point allows me to introduce the issue of the pipeline largely agreeing on their approach, creating an echo chamber. Again, not really sure I want to use the term politics, or just invoke related ideas.
		\switchcolumn*
			
			\subitem Validity (organizational constraints)
		\switchcolumn Of course closely related to reliability, validity allows me to highlight that much of the introduced technology, has not been associated with improvements of safety for refined pipelines (and, by extension, use of natural resources). The motivation for the development and promotion of these technologies hence has to sought elsewhere.
		\switchcolumn*
			
		\item Weaker progress assumption
		\switchcolumn The progress assumption, which allowed the previous transition, is still found here but in a weaker form. Allows me to make in the discussion the point that we cannot expect a "natural" resolution of problems associated with the use of natural resources based on learning.
	\end{itemize}
	\end{paracol}

	\subsection{Hypotheses}
	\begin{paracol}{2}
	\begin{itemize}
		\item Learning from experience hypothesis
		\item Learning from failure experience
		\item Vicarious learning hypothesis
		\item \textit{Maybe}: Industry level learning hypothesis
		\switchcolumn Industry level learning hypothesis--how to test that? Also, better to have a dedicated section on hypotheses or integrate with lit review? Currently, lit review is organized to introduce literature, emphasize knowledge and routines approach. Could a lit review that is laser targeted on the hypotheses work?
		\switchcolumn*
	\end{itemize}
	\end{paracol}