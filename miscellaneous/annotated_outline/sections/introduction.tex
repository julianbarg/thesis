	\section{Introduction}
	
	\begin{paracol}{2}
	\begin{itemize}
		\item Grand challenges: identify, problematize important real-world issues \citep{George2016}
		\switchcolumn "Coopt" grand challenges as my primary motivation--respond to 
		call for more research on grand challenges in management research. 	
		\switchcolumn*
		
		\item Important: resource use \citep{George2015}
		\switchcolumn One grand challenge that was emphasized early on is the analysis of use of natural resources. Claim this space.
		\switchcolumn*
				
		\item Work in this space focusing on ESG metrics right now
			\subitem Give examples	
		\switchcolumn There is not a clear literature that would be discussing resource use. Establish that ESG is the closest thing.
		\switchcolumn*
		
		\item We should enter a level deeper
			\subitem Data as raw and encompassing as possible
		\switchcolumn The ESG data is often criticized for being highly abstract and the outcome of complex social processes. Contrast that with my (better) approach.
		\switchcolumn*
		
			\subitem What is "natural" progression in context, rather than rhetorics
				\subsubitem $\rightarrow$Tease learning?
		\switchcolumn Set up that we want to look at encompassing processes, to gain a comprehensive impression of resource use. Spilling/pollution is exemplary of use (or waste) of resources.
		\switchcolumn*
		
		\item Do this for one example where resource use is as clear as could be
			\subitem Introduction of pipeline industry and available data
		\switchcolumn Present this paper as a study of resource use in one industry, so that we may have a realistic impression of how this unfolds over time.		
	\end{itemize}
	\end{paracol}

	\subsection{Can polluters learn to be clean?}
	\begin{paracol}{2}
	\begin{itemize}
		\item Goal: to appraise pipeline industry's trajectory
		\switchcolumn
		\switchcolumn*
		
		\item Assumption: boundedly rational actors that have an interest to reduce pollution
		\switchcolumn The biggest pushback I expect from an audience which argues that pipeline operators are simply willing to accept pollution and manage the fallout. Maybe the greenwashing paper speaks to this, but this paper assumes that if operators can reduce pollution, they will, and they make reasonable efforts to do so.
		\switchcolumn*
		
		\subitem Introduce BTOF assumptions
		\switchcolumn Use that as an opportunity to communicate with inclined audience that we are moving in a BTOF space.
		\switchcolumn*
		
		\item Use a learning framework to assess their progression
		\switchcolumn Tell the reader that rather than starting from scratch, there is a theoretical framework available which has many of the features available that we need to analyze the empirical context.
		\switchcolumn*		
		
		\item Examples of their learning in technology
		\switchcolumn For anybody who is not yet convinced that learning is a suitable framework for analyzing this context, we can yield anecdotal evidence from the industry that learning is what is going on.
		\switchcolumn*
		
		\subitem \textit{E.g., brief history of pipeline technology?}
		\switchcolumn The disadvantage of introducing the history of pipeline technology is that this goes against the "learning from failure" narrative that I originally wanted to set up.
		\switchcolumn*
		
		\subitem \textit{Or their industry learning curve?}
		\switchcolumn This outline scetches a theoretically motivated path into the paper. Maybe plotted out learning curve of the industry would also be great to convince reader that org learning is applicable.
		\switchcolumn*
		
		\item Limits to learning: pollution continues to be an issue
		\switchcolumn One important finding that I want to tease for the sustainability audience is that learning does not mean that spills will go toward zero. Maybe this could also be a good opportunity to transition and tease more results?
	\end{itemize}
	\end{paracol}



	\subsection{Article structure}
	\begin{paracol}{2}
	\begin{itemize}
		\item Introduce theory on org learning
			\subitem There, will discuss fit between org learning and grand challenges needs
		\switchcolumn[0]*	
			
		\item Mixed methods
			\subitem Qualitative view on specific spills \& learning			
			\subitem Quantitative view on whole industry to understand status quo		
		\switchcolumn Mention \citet{Vergne2012} and \citet{Montgomery2019} here. \citet{Vergne2012} probably closer to my approach.
		\switchcolumn*
		
		
	\end{itemize}
	\end{paracol}
