\section{Conclusion}

Over the last five years, some debate has taken place in the management literature on our use of decreasing natural capital, under the umbrella of grand challenges. This dissertation contributes to that literature, by raising the issue of chemical pollution, and alleviation of impacts. In lieu of a great lot of literature on this issue in the management literature, this dissertation turns to \textit{organizational learning}, a literature that has discussed related problems for some decades.

The purpose of this work is twofold. On the one hand, it brings attention to the great knowledge stock that exists in management research research on organizational learning and the behavior of organizations, which may benefit further discussions on alleviation of impacts, and use of decreasing natural capital in general. On the other hand, by bringing attention to the systemic issues that are highlighted by the grand challenges literature, this dissertation also provides a new direction for research on organizational learning. An empirical analysis of learning on the organization level would have likely overlooked the lack of aggregate learning that exists in parts of the pipeline industry (with regard to refined petroleum pipelines). The grand challenges literature here provides an important impulse to revisit some assumptions.