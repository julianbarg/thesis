\section{Data and Methods}

There are two main sources for data on pipeline spills in the US. (1) The Pipeline and Hazardous Materials Safety Administration (PHMSA) maintains a repository on all pipeline spills that occur. A fairly unique attribute of the data is that there is both qualitative and quantitative data available. Over 300 pipeline spills occur in the United States every year, and more than 100 of them are classified by the PHMSA as significant.\footnote{Meaning an injury, fire, explosion or property damage of over \$50,000, or the spill volume is at least 50bbls. See also \url{https://www.phmsa.dot.gov/sites/phmsa.dot.gov/files/docs/pdmpublic_incident_page_allrpt.pdf} and \url{https://julianbarg.shinyapps.io/incident_dashboard/}, accessed 2020-07-14.} The number of spills is sufficient for quantitative analysis to be sensible, but not at a level where the individual spill becomes meaningless as a unit of analysis. For spills that occurred in 2002 and after, the data quality is generally good, as minor spills typically do not result in fines, but failure to provide information on a spill to PHMSA does carry a fine.\footnote{For instance, up to \$1,000,000 for a failure to immediatly update PHMSA, see \url{https://www.phmsa.dot.gov/sites/phmsa.dot.gov/files/docs/subdoc/3976/gdincidentinstructionsphmsa-f-7100-12014-10-through-2019-04.pdf}, accessed 2020-07-14} PHMSA also provides a dataset on pipeline operators, which allows for identification of the organzation that caused an oil spills, and how many miles of pipelines these organizational operate.\footnote{See \url{https://github.com/julianbarg/oildata}, accessed 2020-07-14}.

(2) The National Transportation Safety Board (NTSB) provides reports on pipeline spills that the agency deems significant. These reports typically have a length of 50-200 pages (usually being more than 100 pages long) and detail the incident, the events leading up to it, and its causes. From 1969 until today, 142 reports and briefs have been published. What makes these reports significant is that NTSB tries to identify these incidents early on, and appear on the scene as early as possible. They spend significant resources to observe the spills as they happen, and investigate the underlying causes (rather than liability) of the spills afterwards.\footnote{For instance, in one case NTSB tried to replicate an error of a SCADA system on a replica of the original SCADA setup \citep{NTSB2002}, and in another case NTSB used various pieces of heavy equipment on a pipeline section to determine what caused the mechanical damages that lead to a spill \citep{NTSB1990}.} On a side note: the NTSB in its reports frequently criticizes the PHMSA.

PHMSA is the primary source for quantitative data, and NTSB the primary source for qualitative data. The NTSB reports are (obviously) biased toward very serious incidents, and the qualitative data available on less serious incidents is much less detailed. Only to some degree, this dearth of information can be counterbalanced by additional research on incidents that are not covered by NTSB, as some incidents are not reported on by any other source except for PHMSA. This lack of information on spills that are perceived as less impactful is a known limitation of research on pipeline spills. Both NTSB and PHMSA also have an overt focus on the direct impact of oil spills. The PHMSA dataset focuses on quantitative attributes of the spill, such as spill volume, volume of recovered oil, and boolean variables on remediation, whereas the NTSB focuses on the immediate impact, such as the magnitude of resulting fires, the number and types of injuries that occurred, and the immediate property damages caused. The PHMSA and NTSB data on the impact needs is supplemented with reports from residents, which both provide an understanding of the impact that the spills have on their lives, and provides a more tacit understanding of the impact on the local ecosystem. And of course the collection of third party data serves to triangulate information.

%There might either be a correlation between the \textit{complexity} of the incident and the \textit{information available} to us that is mediated by the \textit{severity} of the incident, or there could be just a correlation between the \textit{severity} of the incident and the \textit{information available}. In other words, we know whether severe incidents are more complex than less severe ones, but we do still know that they are complex.

The qualitative and quantitative information provides us with three kinds of insights. In the first chapter, I carry out a panel regression to assess the state of organizational and population level learning in the pipeline industry. The qualitative data that is consulted indicates that specific issues are addressed through learning, but new, unique problems keep appearing, which prevents the industry from making further progress. The second chapter will not present any data, but rather discusses the learning literature while only sporadically touching on the phenomenon at hand. The third chapter uses discourse analysis and nonparticipant observation to explore the microfoundations of a coexistence of learning and a bottomed out learning curve.