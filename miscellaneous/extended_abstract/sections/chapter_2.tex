\subsection{Chapter 2: Theoretical Foundations of Learning}

This literature review focuses on organizational learning. The purpose of this chapter is to shed light on the intricacies of learning in the pipeline industry from a theoretical perspective. The first section summarizes the literature of the knowledge-based approach. In particular, the section emphasizes the strength of the knowledge-based approach, which accurately describes the accumulation of a knowledge stock that does take place in an organization when the goal is more or less clear and the environment stable. Further, This first section summarizes some of the other accomplishments of the literature that roughly fall into the knowledge-based stream, specifically its predecessor the learning curve literature, as well as vicarious learning, and population level learning.

The second section summarizes the behavioral approach to organizational learning. In particular, this section highlights the gaps left by the knowledge-based approach that may be filled by work in the behavioral space. The behavioral approach appreciates the nonlinearity and messiness of learning. Significant learning can results from individuals or groups organization that seek for the organization to break with convention \citep{Argyris1978}. This form of learning poses a challenge to the knowledge-based literature, because the changes that take place fall outside the normal rubric of improvement (as exemplified by learning curves). Other examples of these qualitatively different dimensions of learning in the literature are exploration and exploitation \citep{March1991}, and high intellect vs. low intellect learning \citep{March2010}.

Building on the review of the two approaches to learning, this literature review then discusses two critical issues, which become particularly important in the context of pipeline spills. First, in light of the leveling off of the learning curve \citep[which in the pipeline industry takes the shape of a "baseline" of spills, or "normal accidents"][]{Perrow1984}, this section addresses the merits of less complex technologies. The learning literature is implicitly technocentric, but when efficiency does not take the primacy--as is the case when dealing with toxic chemical such as oil-- rather than adding and improving technologies, scaling back and relying on less complex technologies may be a better way to go.

Second, in light of the damages that have been brought about by the petroleum industry, it is  timely to take a critical look at the modernistic assumptions of the learning literature. The learning literature should confront the fact that new technolgies also routinely bring about new problems \citep{Beck1992}. To stay with the theme of this dissertation: pipeline spills (and climate change) are problems that did not exist before the petroleum industry came into being. Acknowledging the conjunction of technology and risks is not meant to take away from the recognition of its merits. Yet, we encourage a debate about the tendency of the literature to equate "newer" with "better" rather than "different".

% multiple learning curves either fall into a learning progress toward a more abstract goal, or in a less modernistic conceptualization, learning [verb] when the organization or population turns its head toward a new goal.