\documentclass{article}

\usepackage{ulem}
\usepackage{natbib}

\title{Ch. 2 Structure}
\author{Julian Barg}

\begin{document}

	\maketitle
	
	\section{Introduction}	
	\begin{itemize}
		\item Grand challenges: identify, problematize important real-world problems \citep{George2016}
		\item Important: resource use \citep{George2015}
	%	\item Call for "fair appraisal" of:
	%		\subitem Raises 8 points on different levels, actors, multilevel
	%		\subitem Skip to solution is incomplete!
	%		\subitem In line with broader literature \citep{Reyers2018}
	%	\newline$\rightarrow$ In conclusion section contrast with "solution-style" work
		\item Work in this space focusing on ESG metrics right now
			\subitem Give examples
		\item We should enter a level deeper
			\subitem Data as raw and encompassing as possible
			\subitem What is natural progression in context, rather than rhetorics
				\subsubitem $\rightarrow$Tease learning?
		\item Do this for one example where resource use is as clear as could be
			\subitem Introduction of pipeline industry and available data
	\end{itemize}

	\subsection{Can polluters learn to be clean?}
	\begin{itemize}
		\item Goal: to appraise pipeline industry's trajectory
		\item Assumption: boundedly rational actors that have an interest to reduce pollution
			\subitem Introduce BTOF assumptions
		\item Use a learning framework to assess their progression
		\item Examples of their learning in technology
			\textit{\subitem E.g., brief history of pipeline technology?
			\subitem Or their industry learning curve?}
		\item Limits to learning: pollution continues to be an issue
	\end{itemize}

	\section{Lit review}
	
	\begin{itemize}
		\item What is required of the appraisal according to \citet{George2016}
			\subitem Raises 8 points on different levels, actors, multilevel
				\subsubitem Articulating and Participating
				\subsubitem Actor Needs and Aspirations
				\subsubitem Societal Barriers
				\subsubitem Organizational Constraints
				\subsubitem Institutional Contexts
				\subsubitem Multilevel Actions
				\subsubitem Coordinating Architectures
				\subsubitem Reinforcing Mechanisms
				\subsubitem Outcomes and Impact
		\item To skip to solution is incomplete!
			\subitem E.g., \citet{Ferraro2015, Slawinski2015}
			\subitem Focus on identify sustainable companies $\rightarrow$ miss industry-wide trends
		\item In line with broader literature \citep{Reyers2018}
	\end{itemize}
	
	\subsection{Organizations acquiring knowledge}
	\begin{itemize}
		\item Learning captures some required elements such as levels and interaction
		\item First stream looks at one outcome variable over time
			\subitem Learning curves/learning from experience
			\subitem Multiple outcome variables would be better, but good start?
		\item Different mechanisms identified on different levels
			\subitem Learning from failure
			\subitem Vicarious learning (institutional context)
			\subitem Industry-level learning (coordinating architecture, multilevel actions)
		\item Build in progression assumption without justification
	\end{itemize}

	\subsection{Organizations developing routines}
	\begin{itemize}
		\item Routines approach
			\subitem Applicable to more tacit dimensions of resource use
			\subitem Less suitable for purely quantitative view
			\subitem But how accurate is the quantitative view anyways?
		\item Capturing more elements
			\subitem Aspirations
			\subitem Politics as barrier
				\subsubitem BTOF \& reliability
			\subitem Validity (organizational constraints)
		\item Weaker progress assumption
	\end{itemize}
	
	\section{Methods}
	\begin{itemize}
		\item Pipeline industry good example of resource use
		\item What could I test
			\subitem Learning from experience
			\subitem Vicarious learning
			\subitem Population level learning
			\subitem Learning from failure
	\end{itemize}
		
	\section{Discussion}
	\begin{itemize}
		\item Learning
			\subitem Describes well what organizations are aspiring in the industry
			\subitem Shows divergence of rhetorics and reality
			\subitem Explains coexistence of stagnation and rhetorics
		\item Complexities (incl. stagnation) captured by substreams of learning
		\item Overall trend is concerning
		\item Tease out dynamics that studies looking for sustainability could have missed
		\item No overall development of pipeline safety, despite individual promising developments
			\subitem This is a system feature, akin to noise/variance in industry
		\item Individual actors making progress could be misleading
		\item New problems arising cancel out solutions found
			\subitem This is contribution to learning
	\end{itemize}
		
	\section{Conclusion}
	\begin{itemize}
		\item Reliability and validity: could look into that for answers/solutions
		\item Limitation: assuming that spills encompassing
	\end{itemize}

\bibliography{../../bibliography}
\bibliographystyle{apalike}

\end{document}