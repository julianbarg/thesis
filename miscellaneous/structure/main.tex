\documentclass{article}

\usepackage{ulem}
\usepackage{hyperref}

\begin{document}

	\section*{Learning in the pipeline industry}

	\subsection*{Proposal structure}
	
	\begin{itemize}
		\item 1980-2000 pipeline spills fell
		\item 2000-2020 refined pipeline spills did not fall
		\item Learning literature (learning curves) predicts this
		---
		\item Learning literature does three things:
			\subitem Improving various performance metrics
			\subitem Vicarious learning
			\subitem Organizational forgetting
		\item Organizational forgetting $\rightarrow$ learning is complex, nonlinear
			\subitem \textit{This deserves more attention in my writing}
		\item Convergence/standstill in learning curves
			\subitem \textit{Use this point to move audience from modern to postmodern ideas}
			\subitem Set up a world where learning is rare
		\item \textit{Organizational knowledge}--how does it fit with standstill
			\subitem Saturation or
			\subitem Forgetting
		\item Mixed methods approach
			\subitem Look into learning standstill
			\subitem Qual: deductive
			\subitem Quant: explain empirical/larger scale observation
				\subsubitem Take advantage of learning from failure
		\item Relevance
			\subitem Highlight converegence stage
			\subitem Emphasize importance of double loop/second order learning/exploration
			\subitem Grand challenges: highlight greater trends		
	\end{itemize}


	\newpage
	\section*{Exemplary article structure}

	Howard-Grenville, J., Nelson, A. J., Earle, A. G., Haack, J. A., \& Young, D. M. (2017). “If Chemists Don’t Do It, Who Is Going To?” Peer-driven Occupational Change and the Emergence of Green Chemistry. In \textit{Administrative Science Quarterly} (Vol. 62, Issue 3). \url{https://doi.org/10.1177/0001839217690530}
	
	\subsection*{Introduction}
	\begin{itemize}
		\item Occupations are crucial for change
		\item Occupational change literature focuses on external triggers
		\item How about change without external trigger?
		\item Give some examples
		\item Internal change--no common enemy--internal heterogeneity as obstacle
		\item Growth of green chemistry and available documents
	\end{itemize}
	
	\subsection*{Lit Review: Occuptions and Occupational change}
	\begin{itemize}
		\item Definition of occupation
		\item Occupation as more than work, "way of life"
		\item Hence, occupations interventions in external imposing processes
	\end{itemize}
	
	\subsubsection*{Occupational Responses to External Triggers for Change}
	\begin{itemize}
		\item Literature: three pathways for occupational change
			\subitem Rules, identity, practice
			\subitem Each discribed in detail
	\end{itemize}
	
	\subsubsection*{Moving Away from Occupational Commonalities}
	\begin{itemize}
		\item Example of occupational heterogeneity
		\item Interpenetration of occupations
		\item Heterogeneity--members experience stuff differently
	\end{itemize}	
	
	\subsection*{Methods}
	
	\subsubsection*{Research Settings}
	\begin{itemize}
		\item Chemistry good example of coherent occupation
		\item Green chemistry is example of internally generated change
		\item Green chemistry '98 book as anecdotal evidence
			\subitem Brief summary
			\subitem Success and failure within occupation
	\end{itemize}
	
	\subsubsection*{Data Collection}
	--
	
	\subsubsection*{Data Analysis}
	--
	
	\subsection*{The Emergence and Evolution of Green Chemistry}
	\begin{itemize}
		\item Early green chemistry raised multiple frames
		\item Different frames resonate with different occupation incumbents
		\item Frame incompatibility trickled down, created more heterogeneity
	\end{itemize}
	
	\subsubsection*{Versatile Framing: Advocates Present Concrete Practices Yet Multiple Distinct Frames}
	\begin{itemize}
		\item Adoption of change in chemistry generally slow
		\item Early adopters efforts to make adoption easy for others
			\item "Versatile framing" strategy with examples
		\item Early adopters offer three different frames
			\item Deliberately loosening control
			\item Normalizing: systematically apply "normal science" model
			\item Moralizing: social benefits, obligation etc.
			\item Pragmatic: efficiency/economic benefits
			\item Different frames resonate with different roles of chemists		
	\end{itemize}

	\subsection*{Role-centric Mobilization: Distinct Frames Resonate with Chemists’ Different Roles}
	\begin{itemize}
		\item Example of advocates applying frames to other chemists roles
	\end{itemize}
	
	\subsubsection*{Normalizing frame resonates with innovator role}
	\begin{itemize}
		\item With colleagues, emphasize chemistry following the scientific method and yielding valuable outcomes
	\end{itemize}
	
	\subsubsection*{Moralizing frame resonates with educator/communicator roles}
	\begin{itemize}
		\item As spokespeople/educators, raising enthusiasm of outsiders/newcomers
	\end{itemize}
	
	\subsubsection*{Pragmatizing frame resonates with problem-solver role}
	\begin{itemize}
		\item In industry/work settings to solve problems in practice
	\end{itemize}

	\subsection*{Experiencing Frame Incompatibility: Tensions of Quality, Commitment, and Complexity}
	\begin{itemize}
		\item Describe diversity of community with examples
		\item Raise notion of incompatibilities/tensions
		\item There are three types of tensions	
	\end{itemize}

	\subsubsection*{(Subsections on tensions)}
	Three tensions, every subsection includes explanation and examples from qual data
	
	\subsection*{Tightening Frames to Reduce Tensions}
	\begin{itemize}
		\item Introducing rigor to ease tensions
		\item Subsuming some goals under other goals
	\end{itemize}

	\subsection*{Sustaining Versatile Framing to Further the Change}
	\begin{itemize}
		\item Agility as a virtue
		\item Different principles for different circumstances, e.g., international
		\item Questioning the potential of overarching tools and metrics
		\item Emphasize importance of integrated view
	\end{itemize}
		
	\subsubsection*{A Model of Peer-driven Occupational Change}
	\begin{itemize}
		\item General model
		\item Versatile frame render possible mobilization of different groups
		\item Versatile frames utilize heterogeneity but give rise to tensions to conflicts between roles
		\item $\rightarrow$ divergent responses, in this case two--tightening frames or sustaining versatile framing
		\item Overall, different from externally triggered change that has been observed before
	\end{itemize}
	
	\subsection*{Discussion}
	\begin{itemize}
		\item External triggers initiate collective response
		\item Internal trigger leverages heterogeneity
		\item Potental for "stable condition of pluralism"
	\end{itemize}
	
	\subsubsection*{Occupational Heterogeneity and Occupational Change}
	\begin{itemize}
		\item Existing literature focuses on homogeneity that allows occupation to respond
		\item This research emphasizes heterogeneity that can generate change
		\item Notion of politics in occupations
		\item This paper: new approach to strategies
			\subitem Not the large professional associations--top-down
			\subitem Individual tailoring of messages, bottom-up
		\item New conceptualization of occupations as loose coalitions
		\item Professionalization may inhibit heterogeneity and hence change
		\item Newcomers may be source of heterogeneity, harbingers of change
		\item When change is externally triggered, heterogeneity can still assist response/adaption
	\end{itemize}
	
	\subsubsection*{Sustained Pluralism and Occupational Change}
	\begin{itemize}
		\item Heterogeneity essential to peer-driven change in occupations
		\item Other work on change argues that ambiguity should be reduced over time
		\item For occupations, convergence unlikely
		\item Also, ambiguity can assist growth
		\item This growth sustains ambiguity but generates tension
		\item Tensions are research opportunity
	\end{itemize}
	
	\subsubsection*{Moralization and Peer-directed Change}
	\begin{itemize}
		\item Occupational change through morals, its difficulty, resisted by experts
		\item Moral calls for change need to be internally generated
		\item Occupations have their own moral frames
		\item Incumbents may be hesitant to raise morals-related stuff
		\item Moral stuff clashes with science/objectivity stuff
	\end{itemize}
	
	\subsubsection*{Boundary condition and Future Research}
	\begin{itemize}
		\item Boundary: observation salient in science, might be different in non-science occupation
		\item Analysis does not cover external environment, e.g., employers
		\item Analysis does not cover the context of external environmental movements
		\item Neat summary in the end
	\end{itemize}
	

\end{document}