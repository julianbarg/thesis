\section{Literatures}

Multiple literatures contribute to this PhD thesis. The challenge will be to make these languages talk to one another. The literature on \textit{Grand Challenges} provides us with a research agenda, but has little to offer (so far) that can guide our work \citep{George2015}. In lieu of input from management research on the use of natural resources, we can draw on works in interdisciplinary journals that discuss use of \textit{natural resources} by organizations \citep[e.g., ][]{Rockstrom2009}. For theory development, this thesis builds on the learning literature. The \textit{learning literature} again falls into two camps: (1) One stream traces back to the research on \textit{learning curves}; this stream looks to disaggregate organizational learning and identify the factors that accelerate or impede learning \citep{Argote2013}. (2) The other stream originates in the \textit{behavioral theory of the firm}; that stream looks at choices and systemic impediments to learning \citep[e.g., ][]{March1963, Levitt1988, Levinthal1993}. Beyond these literatures, \textit{pipeline technology} is discussed in engineering, a stream of sociology conducts \textit{disaster research}, and research on complex systems \citep[especially][]{Perrow1984} also contributes to this thesis. These literatures will play a subordinate role.

\subsection{Organizational learning}

As mentioned above, the literature on organizational learning traces its roots back to two distinct streams of literature that can be distinguished by their definitions of learning. The first stream defines learning as "a change in the organization's knowledge that occurs as a function of experience" \citep[p. 1124]{Argote2011}; we will call this the knowledge-based approach. The second literature holds that organizations learn "by encoding inferences from history into routines that guide behavior" \citep[p. 320]{Levitt1988}; the contributors are much more careful about stating that organizations \textit{know} per se the lessons learned. We will call this second stream of literature the behavioral approach.

\subsubsection{Knowledge-based approach}

The first discussions of organizational learning are found in the learning curve literature \citep{Wright1936}. In particular, WW2 provided a couple of "quasi-experiments"; in the shipbuilding industry, the researchers could observe how with every subsequent unit of production, productivity would improve \citep{Searle1945}. The most straightforward mathematical representation of the learning curve is the progress ratio. For instance, if the progress ratio is \textit{p}, then each time the cumulative output doubles, the unit cost would be predicted to drop to \textit{p}\% of its previous value \citep[p. 15]{Argote2013-1}. In other words, while in the beginning organizational learning allows for a quick reduction of unit cost, eventually, the next doubling of cumulative production is so far out that the unit cost is almost constant. One ambition of the learning curve literature is to mathematically disaggregate learning curves into multiple intraorganizational factors that predict the speed of learning \citep[e.g.,][]{Arrow1962}.

Because so many different factors were found to influence the learning rate, the literature eventually directed its attention to the process of organizational learning itself. A large share of this body of work roughly follows this pattern \citep[taken from][]{Love2014}: the author selects an organization-level performance variable (innovation, measured as sales from newly introduced products), and gathers this data for large companies in an industry or country (Ireland). Then, independent variables are selected that account for the heterogeneity across organizations (innovation linkages, measured as product development with customers or suppliers, joint ventures, etc.). This approach has allowed researchers to identify a broad variety of sources of variation \citep[pp. 18ff]{Argote2013-1}.

A limitation of this relatively formulaic approach however is that it may fail to identify path-breaking innovation. Not all insights are equally important, and the best insights are sometimes difficult to capture with quantitative metrics. For instance, a learning insight might fall outside the regular schema of innovation and lead an organization into a new industry. And many fossil fuel companies are currently (still) successful because they double down on their existing knowledge stock and insulate their industry from changes--an orthodox learning paper might still diagnose learning, if e.g., production increases; but an example of a more interesting question with regard to learning would be which organizations manage to diversify and benefit from the rise of renewable energy.

%The weakness of this approach lies in the lack of its ability to compare different technologies and notice improvements that come about with multiple "generations" of technology or slightly different technologies that serve the same purpose. For instance, to go back to the origin of this literature, we might 

%knowledge transfer, vicarious learning, population level learning

\subsubsection{Behavioral approach}

Rhe Carnegie school early on took notice of organizational learning \citep[e.g., ][]{March1963}. For some time, this literature developed in parallel to the learning curve literature. This difference between the two approaches is best exemplified by \citet{Argyris1978}. \citet{Argyris1978} developed the concept of double-loop learning. The first loop represents adjusting adjustments according to well-known decision criteria, such as launching a promotion when a sales goal is not met. The second loop represents an adjustment of the decision making process itself. For instance, a member of the organization may discover that the organization's goal has become unattainable, and push for a modification of the goal itself. While this literature is generally much more difficult to translate into empirical work, it allows us to talk about issues that fall outside the scope of the learning curve (and knowledge-based) literature. For example, one major criticism of learning curves is that findings may have resulted from a self-fulfilling prophecy--an organization ends up at a certain productivity level \textit{because} that productivity level was the goal. The organization would not overaccomplish, because members lower their efforts when they approach the target. And if the organization falls short of its goal, the organization may move the goal post--by adjusting the goal, or its accounting approach.\footnote{Similarly, the reason why organiztional learning and learning curves appear to be omnipresent could be a result of a publication bias.} The behavioral approach provides us with a language to discuss these issue and similar issues. 

Concepts that speak to the phenomenon of pipeline spills include the aforementioned double-loop learning, exploration and exploitation \citep{March1991}, the competency trap \citep{Levitt1988}, and experiential learning under ambiguity \citep{March1975}. Some streams of the behavioral approach have cross-fertilized the knowledge-based stream. These include work on learning from rare events \citep{March1991b, Maslach2018}, and learning from failure \citep[e.g.,][]{Madsen2010}. These literature talk to some of the tensions that can be observed with regard to pipeline safety--an insistence on existing technology, and a lack of major overhauls, but also surges in activity in response to spills.

%\subsection{Grand Challenges}


