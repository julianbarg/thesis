\section{}

\textit{Natural resources} are still an understudied topic in management research, despite humanity, in some regards, having already overstepped our planet's safe operating boundaries \citep{George2015, Rockstrom2009a}. Most of the challenges humanity faces are related to emissions, which in turn are driven by organizations. A common methodological approach to this topic in management research is to employ ESG indicators, of which there are multiple available. However, a concern with regard to ESG indicators is that different indicators are build on different theorizations, and commensurability is generally low \citep{Chatterji2016}. ESG indicators represent a delicate attempt to condense many different areas of organizational behavior into one value, and authors cannot avoid making judgement calls when deciding on what ESG indicator to use, and how the employ the indicator \citep{Eccles2019}. There is no definitive mathematical key to tallying up poor performance in one area (e.g., carbon emissions) and good performance in another area \citep[e.g., working conditions;][]{Constanza2014}. Research on business sustainability could 

Rather than relying on ESG indicators, for this work we will go the path of picking one area of impact, and taking a much more \textit{in-depth look at one class of pollution}. Specifically, this work will discuss \textit{pipeline spills}. Environmental pollution is one of the nine areas of human activity that was highlighted by \citet{Rockstrom2009} as potentially infringing on our planetary boundaries.\footnote{How close we are to overstepping our planetary boundaries with regard to environmental pollution was not yet assessed my the authors.} The decision to focus on pipeline spills has some advantages and disadvantages. It is easier (albeit sometimes still difficult) to assess and compare the scale of impacts: for each spill, the pipeline operator provides data on commodity, spill volume, and volume recovered. Pipeline spills occur as a number of discrete events every year--not in a scale that would make the individual spill irrelevant and the lot of it a blur, but a sufficiently high number for us to conduct quantitative work, as all most major players in the industry (at least in the US) experiences at least a couple of spills every year. On the other hand, by focusing on pipeline spills, we to some extend "sacrifice" the big picture: while the impacts from pipeline spills are significant, the more important piece are arguably the carbon emissions that pipelines contribute to. This downside may be more than balanced out if the findings generalize to environmental pollution and emissions by organizations in general.

The pipeline industry is a useful context for \textit{studying the "greening" of organizations}. A beneficial attribute of pipeline spills is that the debate is somewhat less entrenched than similar debates in other industries. No side argues that pipeline spills are not harmful.\footnote{The industry only argues that spills can be controlled.} Nor are spills argued to be as a "necessary evil", a downside that cannot be separated from the "good thing" it is attached to. Approaches to pipeline safety do not put into question the business model in general. Further, multiple promising technological approaches to reducing pipeline safety exist, such assmart pigs, SCADA systems, and leak detection systems. Major spills grab the public's attention and are a motivation for pipeline operators to improve performance. One can--at least in theory--have pipelines without having spills. In summary, the pipeline industry should be a great context for us to study what improvements in environmental performance we could expect to see in other industries. Internal and external factors that delay or accelerate greening processes could be studied directly extrapolated without having to worry too much about the dependent variable.


Overall, the pipeline industry has come a far way over the last 40 years. The average annual amount of oil spilled per barrel-mile transported has more than halved (see Figure 1). The safety performance for refined oil has bottomed out since about the turn of the millennium, whereas the crude oil pipeline network might see more improvements.\footnote{Crude oil pipelines are more difficult to operate and maintain, because the crude oil is more corrosive.}

Both operators and critics are right.

%Given this beneficial conditions, the reality of the pipeline industry may come as a bit of a disappointment. Data about oil spills in the United States is provided by the Pipeline and Hazardous Materials Safety Administration (PHMSA).\footnote{PHMSA provides data from 1986 onward, with there being a significant improvement of the quality of the data from 2002 onward and again from 2010 onward.} Data on the pipeline networks of the different operators is available from the PHMSA from 2004 onward. This quantitative data suggests there being limited improvements for the industry as a whole over the last 15 years, despite qualitative data suggesting great strides both in development and deployment of pipeline safety technology. These results are mirrored by data that is somewhat hidden away in a report by the American Petroleum Institute (API).

What can we expect from technology?
