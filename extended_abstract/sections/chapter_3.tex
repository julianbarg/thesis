\subsection{Chapter 3: The Microfoundations of Learning Stagnation}

The first chapter revealed a leveled-off learning curve in the pipeline industry, as far as refined petroleum pipelines are concerned. This third chapter of my dissertation explores the microfoundations of this leveled off learning curve. Somehow, efforts to improve technology and learn from occurring spills coexists with a lack of improvements in pipeline safety on the population level. To explore this phenomenon, this chapter turns to qualitative data.

Two types of data provide access to this phenomenon. (1) A discourse analysis reveals how the industry, at a population level, shields itself from a discovery (by an outside audience) of the lack of improvements. To do so, population level organizations like the American Petroleum Institute (API) emphasize metrics that show the industry in a good light. The regulator also plays an important role in the process. Members of the industry like to emphasize their adherence to the "world class" American regulations to shield themselves from criticism. At the same time, the evaluation of its main regulating body, the Pipeline and Hazardous Materials Safety Administration (PHMSA), depends on the safety performance of the industry. In other words, this regulating body would also suffer from an appearance of the industry as being unsafe. To obtain a positive performance evaluation from its stakeholders--the Department of Transportation, the United States Congress, and ultimately the American public--PHMSA needs to show either that the industry is safe, or (when the industry cannot be presented as safe because high profile incidents occurred in the recent past) that PHMSA and the industry addresses safety problems. In other words PHMSA also has an interest in giving the appearance of learning, over a big-picture analysis of the industry's performance. Finally, NTSB's mission is directly related to learning--the organization was created to analyze transportation-related incidents and define lessons to be learned. The interest of the involved parties in highlighting learning over a big picture analysis of safety performance are explored in a discourse analysis.

Non-participant observation is a potential supplement to the discourse analysis. When the EPA, PHMSA, or coast guard responds to a pipeline spill, over the course of a few days, an important process takes place. Qualitative data indicates that emergency respondents reaction to pipeline spills is often shock. But when the EPA presents its response to pipeline spills (and other pollution events) to the public, the agency emphasizes its ability to carry out environmental remediation. Similarly to the process that the discourse analysis reveals, in the field emergency respondents go through a process of translating an inherently negative event (pipeline spills) into an event where the agency shows positive action. The qualitative data collected for this chapter allows for a further analysis of the microfoundations of a leveled off learning curve, where learning specific lessons coexists with aggregate stagnation.