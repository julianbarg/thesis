\frame{
	\insertsection{Theories}
}

\begin{frame}
	\frametitle{Greenwashing}
	\begin{enumerate}
		\item \citet{Delmas2011}
		\item \citet{Lyon2011}
		\item \citet{Lyon2015}
		\item \citet{Marquis2016}
		\item \citet{Kim2015}
	\end{enumerate}

	\vspace{0.1cm}
	\hrule
	\vspace{0.1cm}
	\small
	Definitions: "any communication that misleads people into adopting overly positive beliefs about an organization’s environmental performance, practices, or products" \citep[p. 226]{Lyon2015}.

	\note{
		\tiny
		\begin{enumerate}
			\item \citet{Delmas2011}--explaining that greenwashing can be cheaper than action. Motivations for greenwashing range from internal communication problems to problematic incentive structures.
			\item \citet{Lyon2011}--economic perspective on greenwashing. Economic rational. Predicting that greenwashing more likely when good environmental performance moderately surprising. Since if its not surprising, firm gains no benefit for having positive performance. And similarly, if it is not surprising, firms gain little from disclosing negative performance. Applicable to my empirical context--it \textit{is} surprising if pipeline operators have persuasive evidence that they are save and clean. And what we see in the empirical context is exactly partial disclosure.
			\item \citet{Lyon2015}--differentiating different kinds of greenwashing. Distinguishing between decoupling and calculated economic greenwashing, and marketing. Providing definition.
			\item \citet{Marquis2016} Role model for large scale greenwashing study. Constructing a variable that captures how much of an organization's environmental disclosure is composed of metrics that actually matter, and how much of it consists of irrelevant metrics. Nice, large scale study, good statistical power \textit{but} the DV is misconstructed.
			\item \citet{Kim2015}--alluding to the fact that a lot more of the communication organizations do is political. Introduces brownwashing: deliberate obfuscation of good environmental performance as not to raise expectations or in case shareholders think anything green is expensive. Overall, gives off the impression that we can take environmental information far less at face value than we thought.
		\end{enumerate}
	}
\end{frame}

\begin{frame}
	\frametitle{Organizational learning}
	\begin{enumerate}
		\item \citet{March1963}
		\item \citet{Argyris1978}
		\item \citet{March1991}
		\item \citet{Argote2013}
		\item \citet{March2010}
	\end{enumerate}

	\note{
		\tiny
	\begin{enumerate}
		\item \citet{March1963} Mostly talking about learning in terms of routine adjustments, but also e.g., of attention and search rules. Important element: emphasizing the political nature of how priorities are set and what is being learned--reliability!
		\item \citet{Argyris1978} Double-loop learning. Learning is more complex than adjustments of inputs and outputs. At some point, fundamental assumptions need to be questioned in order to make an impact--there are a lot of iron tenets in the pipeline industry or fossil fuel in general that are not being touched. Gives the example of changing performance measures that are used to track progress. Maybe pipelines should stop their ridiculous 99.9999\% success rate--now they are focusing on zero accidents--good.
		\item \citet{March1991} I guess Mark may not agree that this should be here as an important paper. But--does a good job of showing that going deeper and deeper on one technology does not hold as much promise as regularly straying far away--exploration!
		\item \citet{Argote2013} Mostly literature review and representative of the knowledge-based view. Which I find problematic, because it fails to recognized the distributed nature of "knowing" in organizations and the interests that clash--politics!
		\item \citet{March2010} Differentiating high-intellect and low-intellect learning. But more importantly, introducing (very briefly) validity and reliability as attributes of knowledge. Knowledge is reliable if it is shared among members of an organization/population. It is valid if it actually helps to bring about its goal/make a difference.
	\end{enumerate}
	}
\end{frame}