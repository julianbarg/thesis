\documentclass[12pt, man, natbib]{apa6}
\usepackage[USenglish]{babel}
\usepackage{setspace}
\usepackage{hyperref}

\title{Why to study pipeline spills, rather than ESG indicators}
\shorttitle{Why to study pipeline spills}
\author{Julian Barg\\barg.julian@gmail.com}
\affiliation{Ivey Business School}
\setcitestyle{authoryear, open={()},close={)},citesep={,},aysep=}


% \abstract{}


\begin{document}
	
	\maketitle
	
	\singlespacing
	
	\section{}	
	
	This section in brief lays out the motivation for studying the pipeline industry. The "gold standard" for sustainability research is to comprehensively measure environmental impacts. A common approach for doing so is to use ESG indicators. However, many barriers have to be overcome to make effective use of ESG indicators. Specifically, how the indicator is constructed has to be taken into consideration: the researcher has to be aware that the indicator is a product of social construction and has to treat it as such when conducting empirical research \citep{Eccles2019}. In particular, comparisons across industries are problematic. ESG indicators are always a combination of other metrics, and when for instance one of these metrics dominates the impact of an industry (e.g., downstream emissions of the fossil fuel industry), that should be taken into consideration during research design. Data availability also tends to be better for large corporations, favoring a cross-industry approach over intra-industry tests.
	
	Moving from an ESG indicator to something more specific means making a sacrifice. The researcher to some degree forgoes the aspiration to measure impacts comprehensively, and research may become susceptible to greenwashing. For example, a chemical producer might try to improve its image by improving worker conditions; to then make any generalized statements on the sustainability of the corporation's operations without also taking into account e.g., environmental emissions would draw a wrong picture. On the flip side, to judge chemical company with excessive deaths only by its environmental impacts would also be flawed. By focusing on just one issue these complexities are lost.
	
	The only context where focusing on just one metric would be justified is when that metric represents the most important area of impact. Coal power plants for instance are characterized by the high number of respiratory problems and indirect deaths they cause through air pollution, and the nuclear industry by its catastrophic potential. For pipelines, the case is less clear, because of their role in the fossil fuel supply chain, and by extension global climate change. However, pipelines are not indispensable for the global fossil fuel. A large share of petroleum transport globally happens by ship, which is also very cost-efficient. Thus, the pipeline industry's environmental impact is largely characterized by pipeline spills, especially because of their catastrophic potential.
	
	I apply data from the context of pipeline operators to two related conceptual areas to show how comprehensive insights  on \textit{organzational learning} and \textit{greenwashing} can be generated from data on the industry. First, we will examine the limits to \textit{organizational learning}. Individual oil spills are well documented, giving us access to the lessons to be learned from hundreds of events every year. In case of the most severe spills, where supposedly the largest amount of learning occurs \citep{Madsen2010}, the spill causes and lessons to be learned are further spelled out in detailed reports by the National Transportation Safety Board (NTSB). \citet{Madsen2009} carries out a similar industry-wide study of organizational learning on fatal accidents in US coal mining. Madsen uses that context to extract evidence on organization-level learning from failure. The context of the pipeline industry allows us to expand on Madsen's work and also comment on the industry-wide convergence of learning (as evident from the normalized rate of spills). Further, the nature of the outcome variable (see previous paragraph) allows for a discussion at the intersection of corporate (environmental) sustainability and organizational learning \citep{George2015}. In other words, this research allows for comments on some of the same processes that were discussed by \citet{Nyberg2017} with regard to climate change being picked up by corporations but insufficiently tackled, albeit from from a slightly different, quantitative perspective. This perspective recognizes changes that have been made while acknowledging that pipeline spills remain to be an important environmental issue.
	
	\textit{Greenwashing} is another area that pipeline spill data can shed light on. As hinted at above, greenwashing is difficult to capture because organizations are deliberately cultivating their image. Where greenwashing takes the shape of decoupling between internal action and communication with the environment \citep{Lyon2015}, ESG indicators are also affected. Pipeline operators can misrepresent their efforts to improve pipeline safety, but pipeline spills are generally well documented. I probably don't have to remind the reader that oil in its crude form is a black, gooey substance. On waterways, refined oil forms distinct films. Both crude and refined oil give off a distinct odor. Pipeline spills are often initially discovered by residents. Emergency responders and specialized spill response staff as well as often journalists are all groups of people that would become aware of pipelines spills once they reach a certain scale. In short there are many reasons why pipeline spills are better documented than e.g., human rights violations or greenhouse gas emissions. For these reasons, we can identify well for the pipeline industry the organizations that communicate a commitment to pipeline safety but exhibit a poor safety performance. Because the datasets are longitudinal and cover (at least for pipeline miles and spills) the complete industry, it offers an opportunity to study the scope of greenwashing in an industry as well as its development over time.
	
	As shown in the two paragraphs above, examining the pipeline industry provides a range of insights. These insights take both the form of contributions to theory development, and the form of taking stock of an industry and its development over time. The second form of insight could become relevant for stakeholders, such as NGOs that monitor the pipeline or other industries. Real-world oriented questions that this research could shed light on are "what should stakeholders expect of polluting industries both in terms of cleanup and greenwashing?" Other questions, this research cannot answer, but give an impetus for an informed debate, e.g., "what level of spills is acceptable, and how do we get there?"
		
\bibliography{bibliography}

\end{document}